% !TeX encoding = UTF-8
% !TeX spellcheck = pl_PL
\documentclass[12pt,a4paper]{article}
\setlength{\emergencystretch}{2em}

\usepackage[marginratio=1:1]{geometry}
\usepackage[T1]{fontenc}
\usepackage{ae,aecompl}
\usepackage[activate={true,nocompatibility},final,tracking=true,kerning=true,spacing=true,stretch=10,shrink=10]{microtype}
\SetTracking{encoding=*, shape=sc}{50}
\microtypecontext{spacing=nonfrench}
\usepackage[utf8]{inputenc}
\usepackage[draft=false]{hyperref}
\usepackage{xcolor}
\usepackage{amssymb}
\usepackage{amsmath}
\usepackage[polish]{babel}
%\usepackage{csquotes}
%\usepackage{tikz-cd}
%\usepackage{float}
%\usepackage{graphicx}
%\usepackage{tikz}
%\usepackage{thmtools}
%\usepackage{thm-restate}
%%\usepackage{mathrsfs}
%\usepackage{enumitem}
%%\usepackage{mathtools}
%%\usepackage[overload]{textcase}
%%\usepackage[numindex,numbib]{tocbibind}
%\setenumerate{label=(\arabic*),topsep=.1\baselineskip,itemsep=-0.1\baselineskip}
%\setitemize{topsep=.1\baselineskip,itemsep=-0.1\baselineskip}
\usepackage[backend=biber,
url=false,
isbn=false,
citestyle=alphabetic,
bibstyle=alphabetic,
autocite=inline,
maxnames=99,
minalphanames=4,
maxalphanames=4,
sorting=nyt,]{biblatex}
\hypersetup{
	colorlinks,
	linkcolor={red!50!black},
	citecolor={blue!50!black},
	urlcolor={blue!80!black}
}

\addbibresource{thesis_bibliography.bib}


\NewBibliographyString{toappear}
\DefineBibliographyStrings{english}{%
	toappear = {to appear},
}
\DefineBibliographyStrings{polish}{%
	toappear = {praca przyjęta},
}
\DefineBibliographyStrings{english}{%
submitted = {submitted},
}
\DefineBibliographyStrings{polish}{%
toappear = {praca wysłana},
}

\title{O pracy ,,Bounded invariant equivalence relations''}
\author{Tomasz Rzepecki}
\date{}




\begin{document}
	\maketitle
	Główną motywacją dla rozprawy doktorskiej jest pytanie o równoważność gładkości (w sensie deskryptywnej teorii mnogości) i typowej definiowalności tzw.\ silnych typów w teorii modeli.
	
	Pierwotna hipoteza o równoważności gładkości i typowej definiowalności silnego typu Lascara, postawiona w pracy \cite{KPS13} (której autorzy, na bazie ogólnych sugestii z wcześniejszej pracy \cite{CLPZ01}, nadali ścisły sens mocy borelowskiej teoriomodelowej grupy Galois i silnego typu Lascara), została udowodniona w pracy \cite{KMS14}, a następnie rozszerzona na tzw.\ silne typy orbitalne $F_\sigma$ w pracach \cite{KM14} i \cite{KR16} (z których ta druga jest oparta na mojej pracy magisterskiej).
	
	W obydwu pracach pozostały otwarte pytania o możliwość rozszerzenia równoważności na szersze klasy silnych typów.
	
	Główne wyniki rozprawy doktorskiej pochodzą z trzech prac \cite{KPR15}, \cite{Rz16} i \cite{KR18}, które łącznie dają w zasadzie optymalny opis sytuacji, w których wspomniana równoważność zachodzi (w szczególności obejmując wszystkie silne typy zadane na zbiorze realizacji jednego typu zupełnego i wszystkie orbitalne silne typy).
	
%	\begin{itemize}
%		\item
%		\cite{KPR15} (wspólnej z Krzysztofem Krupińskim i Anandem Pillayem),% w której, korzystając z dynamiki topologicznej dla grupy automorfizmów tzw.\ modelu monstrum, udowodniliśmy między innymi pełną równoważność gładkości i typowej definiowalności dla typów zadanych na zbiorze realizacji jednego typu zupełnego nad zbiorem pustym (tj.\ na jednej orbicie grupy automorfizmów),
%		\item 
%		\cite{Rz16} (mojej własnej),% której głównym wynikiem było znalezienie szerszego kontekstu tzw.\ słabo orbitalnych relacji równoważności (w tym tych rozważanych w \cite{KPR15}), dla których --- wykorzystując \cite{KPR15} --- również można pokazać podobną równoważność, oraz
%		\item 
%		\cite{KR18} (wspólnej z Krzysztofem Krupińskim).% w której rozwinęliśmy metody użyte w pracy \cite{KPR15}, aby przedstawić teoriomodelową grupę Galois, a także zbiór klas dowolnego typu zadanego na zbiorze realizacji jednego typu zupełnego, w postaci ilorazu Galois przez podgrupę, w sensie topologicznym, algebraicznym (w przypadku grupy Galois), a także do pewnego stopnia w sensie mocy borelowskiej. W szczególności przy założeniu NIP uzyskaliśmy pełną równość mocy borelowskiej grupy Galois (czy też silnego typu) i odpowiedniego ilorazu grupy polskiej.
%	\end{itemize}
	%W przeciwieństwie do wspomnianych powyżej prac, w rozprawie doktorskiej rozważam bardziej abstrakcyjny, czysto topologiczny (czy też dynamiczno topologiczny) kontekst. %, a następnie dowodzę główne twierdzenia nie odwołując się wprost do teoriomodelowych motywacji.
	W rozprawie doktorskiej wyabstrahowałem kilka stosunkowo naturalnych własności relacji równoważności w układach dynamicznych występujących w teorii modeli, które były wykorzystywane w owych pracach. To pozwoliło na udowodnienie analogicznych twierdzeń w tym właśnie abstrakcyjnym kontekście, nie odwołując się wprost do żadnych teoriomodelowych faktów czy założeń, bazując na dynamice topologicznej (i jej związkach z dychotomią Bourgaina-Fremlina-Talagranda), własnościach grup zwartych i deskryptywnej teorii mnogości.
	
	Następnie, korzystając z owych abstrakcyjnych twierdzeń, wnioskujemy (głównie przez stosunkowo nietrudne sprawdzenie aksjomatów) wszystkie wyniki z prac \cite{KPR15}, \cite{Rz16} i \cite{KR18}, część z nich nawet nieco wzmacniając. Wydaje się też, że w podobny sposób można z nich wywnioskować podobne fakty w zasadniczo wszystkich teoriomodelowych sytuacjach, gdzie można się takich wyników spodziewać, na co podaję przykłady.
	
	Oprócz równoważności gładkości i typowej definiowalności (wyrażonej w rozprawie w postaci eleganckiej trychotomii) głównym wynikiem jest przedstawienie pewnych przestrzeni silnych typów, a w szczególności teoriomodelowej grupy Galois, jako ilorazu zwartej grupy polskiej przez podgrupę, w ogólności jako przestrzeni topologicznej z działaniem grupy (a dla grupy Galois również jako grupy topologicznej), a przy założeniu NIP również w sensie mocy borelowskiej.
	
	To pozwala na dokładne wyliczenie mocy borelowskiej grupy Galois, silnego typu Lascara i ogólnie dowolnego silnego typu, poprzez wyznaczenie odpowiedniej grupy polskiej wraz z podgrupą, przez którą należy ją podzielić. W pracy podaję przykład, jak w ten sposób wyznaczyć grupę Galois (wraz z mocą borelowską) w konkretnych przykładach (rozważanych wcześniej w \cite{CLPZ01} i \cite{KPS13}). Uzyskujemy też częściową odpowiedź na postawione w pracy \cite{KPS13} pytanie o możliwą moc borelowską silnego typu Lascara i grupy Galois: przy założeniu NIP jest ona równa mocy borelowskiej ilorazu zwartej grupy polskiej przez podgrupę $F_\sigma$. Można też oczekiwać, że uzyskane wyniki będą pomocne w odpowiedzi na inne pytania postawione w \cite{KPS13} (np.\ o monotoniczność mocy borelowskiej silnego typu Lascara), a także inne pokrewne (np.\ o możliwe moce borelowskie dowolnych silnych typów).
	\AtNextBibliography{\small}
	\vspace{-1em}
	\printbibliography
\end{document}
