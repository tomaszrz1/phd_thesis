
	\chapter{(Weakly) group-like equivalence relations in model theory and beyond}
	\chaptermark{Group-like equivalence relations in particular contexts}
	\label{chap:applications}
	In this chapter, we apply the main results of Chapter~\ref{chap:grouplike} in specific contexts, mostly model-theoretic. In particular, we prove Main~Theorems~\ref{mainthm_group_types}, \ref{mainthm:smt} and \ref{mainthm:nwg}.
	\section{Compact group actions}
	\begin{ex}
		\label{ex:cpct_glike}
		\begin{figure}[H]
			\begin{tikzcd}
				G \arrow[r]\arrow[dr]& EL=G\arrow[d,"R"]\arrow[dr,"r"]& \\
				\tilde G=G \arrow[r] & X=G \arrow[r]& G/N
			\end{tikzcd}
		\end{figure}
		Suppose $G$ is a compact Hausdorff group and consider $X=G$ with $G$ acting by left translations, with $x_0=e\in G$. Then for any normal $N\unlhd G$ the relation $E=E_N$ of lying in the same $N$-coset is properly group-like on $G$, with ${\equiv}$ on $\tilde G$ being just the equality relation.
		
		Pseudocompleteness means just that whenever $g_i\to x_1$, $h_i\to x_2$ and $g_ih_i\to x_3$ for some nets $(g_i)_i,(h_i)_i$ in $G$ and $x_0,x_1,x_2\in G$, then $x_1x_2=x_3$. But this is clear by joint continuity. Other axioms are easy to verify.
		
		Furthermore, $E_N$ is also uniformly properly group-like: indeed, we can take for $\mathcal E$ to be just the family of sets of the form $D_A:=\{(g,hg)\mid g\in G, h\in A \}$, where $A$ ranges over finite symmetric subsets of $N$ containing the identity. Each $D_A$ is closed (as the continuous image of $G\times A$ by multiplication, which is closed by compactness) and $(D_A)'=D_{A\cdot A}$ has the properties required by Definition~\ref{dfn:unif_prop_glike}.
		
		The conclusion of Lemma~\ref{lem:main_abstract_grouplike} is trivial in this case: the map $G\to G^G$ ($g\mapsto \pi_{G,g}$) is continuous and injective, and therefore an embedding into a closed subset, so $EL=G$, and the only idempotent is the identity in $G$, and thus $u\cM=G$, while $H(u\cM)=\{e\}$.
		
		Likewise, the conclusion of Corollary~\ref{cor:metr_smt_cls} reduces to Proposition~\ref{prop:trichotomy_for_groups}.
		\xqed{\lozenge}
	\end{ex}
	
	\begin{ex}
		Consider a transitive action of a compact Hausdorff group $G$ on a compact Hausdorff space $X$, and let $E$ be any $G$-invariant equivalence relation on $X$. Then $E$ is dominated by equality on $G$, which is clearly closed group-like (and also properly uniformly group-like), so $E$ is weakly closed group-like. By applying Theorem~\ref{thm:main_abstract}, we recover most of Proposition~\ref{prop:toy_main} (the rest can be essentially recovered via Remark~\ref{rem:strengthening}, although this reasoning is circular), and by applying Corollary~\ref{cor:metr_smt_cls}, we recover Corollary~\ref{cor:toy_trich}.
		\xqed{\lozenge}
	\end{ex}
	
	\section{Automorphism group actions}
	In this section, we will be looking at dynamical systems stemming from automorphism group actions, of the form $(\Aut(M),S_m(M),\tp(m/M))$ (and other similar ones). In this context, we will find the naturally occurring (weakly) group-like equivalence relations and then apply to them what we learned from Chapter~\ref{chap:grouplike}. Thus we will recover (with some improvements) the main results of the papers \cite{KPR15} (joint with Krzysztof Krupiński and Anand Pillay) and \cite{KR18} (joint with Krzysztof Krupiński).
	
	\subsection*{Lemmas}
	We intend to apply results of Chapter~\ref{chap:grouplike}. In the following lemmas, we will show that their hypotheses are satisfied in the case of actions of automorphism groups on certain type spaces.
	
	\begin{lem}[pseudocompleteness for automorphism groups]
		\label{lem:pseudocompleteness_for_aut}
		Suppose $M$ is a model and $a$ is an arbitrary tuple in $M$ (e.g.\ an enumeration of $M$), while $N\succeq M$ is an $\lvert M\rvert^+$-saturated and $\lvert M\rvert^+$-strongly homogeneous model (for example, $N=\fC$ and $M$ is small in $\fC$).
		
		Then whenever $(\sigma_i)_i$ and $(p_i)_i$ are nets in $\Aut(M)$ and $S_a(M)$ (respectively) such that $\tp(\sigma_i(a)/M)\to q_1$, $p_i\to q_2$ and $\sigma_i(p_i)\to q_3$ for some $q_1,q_2,q_3\in S_a(M)$, there are $\sigma'_1,\sigma'_2\in \Aut(N)$ such that $\tp(\sigma'_1(a)/M)=q_1$, $\tp(\sigma'_2(a)/M)=q_2$ and $\tp(\sigma'_1\sigma'_2(a)/M)=q_3$. (This is pseudocompleteness for $\tilde G=\Aut(N)$, $X=S_a(M)$ and the map $\tilde G\to S_a(M)$ given by $\sigma\mapsto \tp(\sigma(a)/M)$, see Definition~\ref{dfn:prop_glike}.)
	\end{lem}
	\begin{proof}
		Let for each $i$, choose $b_i\in N$ such that $b_i\models p_i$ (so in particular, $b_i\equiv a$), and extend $\sigma_i$ to $\bar\sigma_i\in \Aut(N)$. Then by the assumptions, for every $\varphi_1(x),\varphi_2(x),\varphi_3(x)$ in $q_1,q_2$ and $q_3$ (resp.) we have, for sufficiently large $i$, $N\models \varphi_1(\bar \sigma_i(a))\land \varphi_2(b_i)\land \varphi_3(\bar\sigma_i(b_i))\land a\equiv b_i\land ab_i\equiv \bar\sigma_i(a)\bar\sigma_i(b_i)$. Now by compactness, we have $a_1,a_2,a_3\in N$ such that $N\models q_1(a_1)\land q_2(a_2)\land q_3(a_3)\land a\equiv a_2\land aa_2\equiv a_1a_3$. Any $\sigma'_1, \sigma'_2\in \Aut(N)$ such that $\sigma'_2(a)=a_2$, $\sigma'_1(aa_2)=a_1a_3$ satisfy the conclusion of the lemma.
	\end{proof}
	
	\begin{lem}
		\label{lem:F_0_for_aut}
		Suppose $A\subseteq \fC$ is small, enumerated by the tuple $a$, while $Y$ is type-definable over $A$.
		
		Then the set $F_0=\{\tp((\sigma'_1)^{-1}\sigma'_2(a)/A) \mid \sigma'_1,\sigma_2'\in \Aut(\fC)\land \sigma'_1(a)\equiv_A \sigma'_2(a)\in Y_A \}$ is closed in $S_a(A)$.
	\end{lem}
	\begin{proof}
		Let $F_1$ be the set of $\tp(b/A)$ such that $\fC\models(\exists a_1,a_2) aa_1\equiv aa_2\land a\equiv a_2\land ab\equiv a_1a_2\land a_1\in Y$. Clearly, $F_1$ is a closed subset of $S_a(A)$. We will show that $F_0=F_1$.
		
		For $\subseteq$, take some $p\in F_0$, as witnessed by $\sigma'_1,\sigma'_2\in \Aut(\fC)$ and take $b=(\sigma'_1)^{-1}\sigma'_2(a)$, $a_1=\sigma'_1(a)$ and $a_2=\sigma'_2(a)$. It is easy to check that they witness that $\tp(b/A)\in F_1$.
		
		For $\supseteq$, take some $p\in F_1$, witnessed by $b,a_1,a_2$. Take $\sigma_1',\sigma'_2\in \Aut(\fC)$, such that $\sigma'_1(ab)=a_1a_2$ and $\sigma'_2(a)=a_2$. It is easy to check that these $\sigma'_1,\sigma'_2$ witness that $p=\tp(b/A)\in F_0$.
	\end{proof}
	
	When reading the proof of Lemma~\ref{lem:lascar_grouplike}, it may be useful to compare the diagram below to the diagram in Definition~\ref{dfn:prop_glike}.
	\begin{center}
		\begin{tikzcd}
			\Aut(M) \ar[r] \ar[dr] & E(\Aut(M), S_m(M)) \ar[d,"R"]\ar[dr,"r"]\\
			\Aut(\fC)\ar[r] & S_m(M)\ar[r] & S_m(M)/{\equiv_\Lasc^M}=\Gal(T)
		\end{tikzcd}
	\end{center}
	\begin{lem}
		\label{lem:lascar_grouplike}
		Let $M$ be a small ambitious model (see Definition~\ref{dfn:ambitious_model}), enumerated by $m$. Consider the $\Aut(M)$-ambit $(\Aut(M),S_m(M),\tp(m/M))$. Then $E={\equiv_\Lasc^M}$ is a uniformly properly group-like equivalence relation on $S_m(M)$.
		
		More generally, if $G^Y\leq \Gal(T)$ is closed, and $M$ is ambitious relative to $G^Y$ (see Definition~\ref{dfn:ambitious_model}), then $(G^Y(M),Y'_M,\tp(m/M))$ (where $Y'$ is as there) is an ambit and ${\equiv_\Lasc^M}\restr_{Y'_M}$ is a uniformly properly group-like equivalence relation on it.
	\end{lem}
	\begin{proof}
		Notice that the ``base" case follows from the ``moreover" case simply by taking $G^Y=\Gal(T)$, so we will treat the second case.
		
		Note that $Y'$ is type-definable over $M$ immediately by Fact~\ref{fct: characterization of topology on Gal_L(T)}. The fact that $(G^Y(M),Y'_M,\tp(m/M))$ is an ambit follows immediately by definition of a relatively ambitious model
		
		Note that almost immediately by our assumptions, $Y'_M$ is the preimage of $G^Y$ by the quotient map $S_m(M)\to \Gal(T)$ from Fact~\ref{fct:sm_to_gal}, so we may identify $G^Y$ and $Y'_M/{\equiv_\Lasc^M}$, and ${\equiv_\Lasc^M}\restr_{Y'_M}$ is evidently group-like.
		
		Let $\tilde G=\{\sigma'\in \Aut(\fC)\mid \sigma'\Autf(\fC)\in G^Y \} $. Immediately by the definitions, $\tilde G$ maps onto $Y'_M$ via $\sigma'\mapsto \tp(\sigma'(m)/M)$, and the induced map $\tilde G\to G^Y$ is just $\sigma'\mapsto \sigma'\Autf(\fC)$, which is of course a homomorphism. By applying Lemma~\ref{lem:pseudocompleteness_for_aut}, we obtain pseudocompleteness (in the sense of Definition~\ref{dfn:prop_glike}) --- to see that, just notice that if $\sigma_i\in G^Y(M)$ and $p_i\in Y'_M$, then (since $Y'_M$ is closed), the $q_1,q_2,q_3$ as in Lemma~\ref{lem:pseudocompleteness_for_aut} are in $Y'_M$, so $\sigma_1',\sigma_2'\in \tilde G$.
		
		By Lemma~\ref{lem:F_0_for_aut} (applied to $Y=Y'$ and $A=M$), we conclude that ${\equiv_\Lasc^M}\restr_{Y'_M}$ is properly group-like.
		
		To see that it is uniformly properly group-like, let $\mathcal E=\{F_n\mid n\in\bN\}$, where $F_n$ is the set of pairs of types $p_1,p_2\in Y'_M$ such that there exist some $m_1,m_2$ satisfying $p_1$ and $p_2$ (respectively) and such that $d_\Lasc(m_1,m_2)\leq n$ (cf.\ Definition~\ref{dfn:Lascar distance}). Clearly, each $F_n$ is symmetric, reflexive and by Fact~\ref{fct:distance_tdf}, they are all closed in $S_m(M)^2$. We will show that $(F_n)':=F_{2n+2}$ has the properties postulated in Definition~\ref{dfn:unif_prop_glike}.
		
		Indeed, if $d_\Lasc(a,b_1)\leq n$ and $d_\Lasc(b_2,c)\leq n$ and $b_1\equiv_M b_2$, then $d_\Lasc(b_1,b_2)\leq 1$, so by triangle inequality $d_\Lasc(a,c)\leq 2n+1$, so $F_n\circ F_n\subseteq F_{2n+1}\subseteq F_{2n+2}$. On the other hand, and if $(\tp(m/M),\tp(\sigma(m)/M))\in F_n$, then there are some $\sigma_1,\sigma_2\in \Aut(\fC/M)$ such that $d_\Lasc(\sigma_1(m),\sigma_2\sigma(m))\leq n$, so $d_\Lasc(m,\sigma_1^{-1}\sigma_2\sigma(m))\leq n$, so by Fact~\ref{fct:diameter_witnessed_by_model}, $d_\Lasc(\sigma_1^{-1}\sigma_2\sigma(m))\leq n+1$, and hence (because $\sigma_1^{-1}\sigma_2$ fixes $M$ pointwise) $d_\Lasc(\sigma)\leq n+2$, i.e.\ for every $a$ we have $d_\Lasc(a,\sigma(a))\leq n+2$, in particular, for every $\sigma'\in \Aut(\fC)$ we have $(\tp(\sigma'(m)/M),\tp(\sigma\sigma'(m)/M))\in F_{n+2}\subseteq F_{2n+2}$. Finally, from Fact~\ref{fct:Lascar_equivalent}, it follows easily that ${\equiv_\Lasc^M}=\bigcup_n F_n$, which completes the proof.
	\end{proof}
	
	
	\begin{lem}
		\label{lem:lascar_dominates}
		Given any strong type $E$ on $p(\fC)$, where $p\in S(\emptyset)$, and a small model $M$ enumerated by $m$, such that some $a\subseteq m$ realises $p$, the relation $\equiv_\Lasc^M$ on $S_m(M)$ dominates the relation $E^M$ on $S_a(M)$ (via the restriction $S_m(M)\to S_a(M)$).
		
		More generally, if $Y$ is any $\equiv_\Lasc$-invariant, type-definable set containing $a$, such that $\Aut(\fC/\{Y\})$ acts transitively on $Y$, if $M$ is a small model enumerated by $m\supseteq a$, ambitious relative to $G^Y=\Aut(\fC/\{Y\})/\Autf(\fC)$ (cf.\ Definition~\ref{dfn:ambitious_model}), then $(G^Y(M),Y_M,\tp(a/M))$ is an ambit, and ${\equiv_\Lasc^M}\restr_{Y'_M}$ dominates $E^M\restr_{Y_M}$, where $Y'=\Aut(\fC/\{Y\})\cdot m$.
		
		In particular (by Lemma~\ref{lem:lascar_grouplike}), $E^M$ and $E^M\restr_{Y_M}$ are weakly uniformly properly group-like.
	\end{lem}
	\begin{proof}
		For the first part, just note that the relation $\equiv_\Lasc^M$ on $S_a(M)$ is a refinement of $E^M$, and $\equiv_\Lasc^M$ on $S_m(M)$ dominates $\equiv_\Lasc^M$ on $S_a(M)$ via the restriction mapping $S_m(M)\to S_a(M)$, which follows from the trivial observation that if two tuples are $\equiv_\Lasc$-equivalent, then their subtuples (chosen from corresponding coordinates) are also $\equiv_\Lasc$-equivalent. Since $\Gal(T)=S_m(M)/{\equiv_\Lasc^M}$ acts on $p(\fC)/E$ (and the map $S_m(M)/{\equiv_\Lasc^M}\to S_a(M)/E^M$ is equivariant), this shows that $\equiv_\Lasc^M$ dominates $E^M$.
		
		For the second part, apply Fact~\ref{fct: characterization of topology on Gal_L(T)} and Lemma~\ref{lem:lascar_grouplike} to conclude that $(G^Y(M),Y'_M,\tp(m/M))$ is an ambit and ${\equiv_\Lasc^M}\restr_{Y'_M}$ is uniformly properly group-like. Now, note that the restriction map $S_m(M)\to S_a(M)$, induces an ambit morphism $(\Aut(M),Y'_M,\tp(m/M))\to (\Aut(M),Y_M,\tp(a/M))$. The domination of $E\restr_{Y_M}$ by ${\equiv_\Lasc^M}\restr_{Y'_M}$ follows the same as in the preceding paragraph.
	\end{proof}
	
	\subsection*{Results for automorphism groups}
	\begin{rem}
		\label{rem:wglike_closedness_to_typdef}
		Immediately by Lemma~\ref{lem:lascar_dominates}, we can use Proposition~\ref{prop:top_props_of_wglike} to recover Proposition~\ref{prop:type-definability_of_relations}. We also deduce that if $p\in S(\emptyset)$ and $E$ is an arbitrary strong type on $X=p(\fC)$, then $X/E$ is an $R_1$ space (which was \cite[Proposition 1.12]{KPR15}, but only for $R_0$).\xqed{\lozenge}
	\end{rem}
	
	
	The following theorem is Main~Theorem~\ref{mainthm:nwg} and it generalises Fact~\ref{fct:newelski}. It appeared as \cite[Theorem 5.1]{KPR15} (joint with Krzysztof Krupiński and Anand Pillay). Here, we deduce it from the abstract Theorems~\ref{thm:general_cardinality_intransitive} and \ref{thm:general_cardinality_transitive}. Note that --- in contrast to Theorem~\ref{thm:main_aut} below --- we do not require the language to be countable.
	\begin{thm}
		\label{thm:nwg}
		Suppose $E$ is an analytic strong type defined on $X=p(\fC)$ for some $p\in S(\emptyset)$, and $Y\subseteq X$ is type-definable and $E$-saturated. Suppose $\lvert Y/E\rvert<2^{\aleph_0}$.
		
		Then $E$ is type-definable (note that by Remark~\ref{rem:tdf_iff_restr}, this is equivalent to $E\restr_Y$ being type-definable), and if, in addition, $\Aut(\fC/\{Y\})$ acts transitively on $Y/E$, then $E\restr_Y$ is relatively-definable (as a subset of $Y^2$).
	\end{thm}
	\begin{proof}
		Recall from Fact~\ref{fct: Borel in various senses} that $E\restr_Y$ is relatively definable or type-definable if and only if $E^M\restr_{Y_M}$ is clopen or closed (respectively) for some (equivalently, every) small model $M$.
		
		Now, if $M$ is any ambitious model (which exists by Proposition~\ref{prop:amb_exist}), then we can just apply Lemma~\ref{lem:lascar_dominates} and then Theorem~\ref{thm:general_cardinality_intransitive} to conclude that if $Y/E=Y_M/{E^M}$ has cardinality less than $2^{\aleph_0}$, then $E^M\restr_{Y_M}$ is closed, so $E$ is type-definable.
		
		If $\Aut(\fC/\{Y\})$ acts transitively on $Y/E$, then again by Proposition~\ref{prop:amb_exist}, we can choose $M$ to be ambitious relative to $G^Y=\Aut(\fC/\{Y\})/\Autf(\fC)$. Since $Y$ is type-definable, it follows from Fact~\ref{fct: characterization of topology on Gal_L(T)} that $G^Y$ is closed. Thus, we can just apply Lemma~\ref{lem:lascar_dominates} followed by Theorem~\ref{thm:general_cardinality_transitive}.
	\end{proof}
	
	
	The following is one of the main results of the thesis, and is a part of Main~Theorem~\ref{mainthm_group_types}. Most of it is \cite[Theorem 8.1]{KR18} (joint with Krzysztof Krupiński). Compared to it, we relax the global NIP assumption for the ``furthermore'' part: we assume only that $Y$ has NIP.
	\begin{thm}
		\label{thm:main_aut}
		Suppose that the theory is countable.
		
		Suppose $E$ is a strong type defined on a type-definable set $X$ (in a countable product of sorts), and $Y\subseteq X$ is type-definable, $E$-saturated and such that $\Aut(\fC/\{Y\})$ acts transitively on $Y$ (e.g.\ $Y=[a]_{\equiv}$ of $Y=[a]_{\equiv_\KP}$ for some tuple $a$). Choose $a\in Y$.
		
		Then there is a compact Polish group $\hat G^Y$ acting continuously on $Y/E$, and such that the stabiliser $H$ of $[a]_E$, and the orbit map $\hat r_Y\colon \hat G^Y\to Y/E$, $\hat g\mapsto \hat g\cdot [a]_E$ have the following properties:
		\begin{enumerate}
			\item
			$H\leq \hat G^Y$ and fibres of $\hat r_Y$ are exactly the left cosets of $H$ (so $\hat G^Y/E|_{\hat G^Y}=\hat G^Y/H$),
			\item
			$\hat r_Y$ is a topological quotient map (so it induces a homeomorphism of $\hat G^Y/H$ and $X/E$),
			\item
			$E\restr_Y$ is relatively definable or type-definable if and only if $H$ is clopen or closed (respectively)
			\item
			if $E$ is $F_\sigma$, Borel, or analytic (respectively), then so is $H$,
			\item
			$\hat G^Y/H\leq_B E$.
		\end{enumerate}
		Furthermore, if $Y$ has NIP (in particular, if $T$ has NIP), then $\hat G^Y/H\sim_B E$.
	\end{thm}
	\begin{proof}
		By Proposition~\ref{prop:amb_exist}, we can find some $M$ which is ambitious relative to $G^Y=\Aut(\fC/\{Y\})/\Autf(\fC)$. Then by Lemma~\ref{lem:lascar_dominates}, $E^M\restr_{Y_M}$ is weakly uniformly properly group-like (in the ambit $(G^Y(M),Y_M,\tp(a/M))$), so Theorem~\ref{thm:main_abstract} applies to $E^M\restr_{Y_M}$.
		
		Now, if $Y$ has NIP, then by Corollary~\ref{cor:NIP_implies_tame}, $(G^Y(M), Y_M)$ is tame, so in particular,  so we also have (6) of Theorem~\ref{thm:main_abstract}.
		
		To complete the proof, just note that:
		\begin{itemize}
			\item
			by definition, the Borel cardinality of $E\restr_Y$ is the Borel cardinality of $E^M\restr_{Y_M}$,
			\item
			we identify $Y/E$ and $Y_M/{E^M}$ (via the natural homeomorphism),
			\item
			by Fact~\ref{fct: Borel in various senses}, $E\restr_Y$ is relatively definable, type-definable, $F_\sigma$, Borel, analytic if and only if $E^M\restr_{Y_M}$ is clopen, closed, $F_\sigma$, Borel or analytic (respectively).
		\end{itemize}
		These observations allow us to translate the conclusion of Theorem~\ref{thm:main_abstract} into the conclusion of the theorem, and thus we are done.
	\end{proof}
	
	\begin{rem}
		By referring back to the statement of Theorem~\ref{thm:main_abstract}, we can see that in Theorem~\ref{thm:main_aut}, the group $\hat G^Y$ is $u\cM/H(u\cM)\Core(D)$ for the ambit $(G(M),Y_M,y_0)$, and the action is induced by the action of $E(G(M),Y_M)$ on $Y_M$.\xqed{\lozenge}
	\end{rem}
	
	\begin{prop}
		\label{prop:orbital_in_main_aut}
		In Theorem~\ref{thm:main_aut}, if the stabiliser of $[a]_E$ is a normal subgroup of $\Aut(\fC/\{Y\})$, then $H$ is a normal subgroup of $\hat G^Y$, and $\hat r_Y$ is a topological group quotient mapping onto then $Y/E$ (equipped with the group structure obtained by identification with $\Aut(\fC/\{Y\})/\Stab_{\Aut(\fC/\{Y\})}\{[a]_E\}$).
	\end{prop}
	\begin{proof}
		It is not hard to see that under the hypotheses, the map $G^Y\to Y/E$ given by $\sigma\Autf(\fC)\mapsto \sigma \Stab_{\Aut(\fC/\{Y\})}\{[a]_E\}$ is a homomorphism. But $G^Y$ is naturally identified with $Y'_M/{\equiv_\Lasc^M}$ (with $Y'$  chosen as in the proof of Lemma~\ref{lem:lascar_dominates}, so that ${\equiv_\Lasc^M}$ is uniformly properly group-like on $Y'_M$ and dominates $E^M$). Hence, the assumptions of Proposition~\ref{prop:wgl_homom} are satisfied. Thus Theorem~\ref{thm:main_abstract}(7) applies, and we are done.
	\end{proof}
	
	\begin{cor}
		\label{cor:galois_quotient}
		The Galois group $\Gal(T)$ of any countable first order theory $T$ is isomorphic as a topological group to the quotient of a compact Polish group by an $F_\sigma$ subgroup.
		
		If the theory is NIP, it also has the same Borel cardinality as that quotient.
				
		The same is true for $\Gal_0(T)$.
	\end{cor}
	\begin{proof}
		Choose any tuple $m$ enumerating a model and apply Theorem~\ref{thm:main_aut} to $Y=[m]_{\equiv}$ (for $\Gal(T)$) and $Y=[m]_{\equiv_\KP}$ (for $\Gal_0(T)$), $a=m$ and $E={\equiv_\Lasc}$, noting that Proposition~\ref{prop:orbital_in_main_aut} applies (e.g.\ because by Fact~\ref{fct:diameter_witnessed_by_model}, the relevant stabiliser is just $\Autf(\fC)$).
	\end{proof}
	
	The following trichotomy appeared (essentially) as \cite[Corollary 6.1]{KPR15} (joint with Krzysztof Krupiński and Anand Pillay). It constitutes most of Main~Theorem~\ref{mainthm:smt} (completed by Corollary~\ref{cor:smt_type}).
	\begin{cor}
		\label{cor:trich_plus}
		Suppose that the theory is countable, while $E$ is a strong type (on a set of countable tuples), and $Y$ is type-definable, $E$-saturated, and such that $\Aut(\fC/\{Y\})$ acts transitively on $Y$ (e.g.\ $Y\in \{[a]_\equiv,[a]_{\equiv_{\textrm{Sh}}},[a]_{\equiv_\KP} \}$ for some countable tuple $a$). Then exactly one of the following is true:
		\begin{enumerate}
			\item
			$E\restr_Y$ is relatively definable (as a subset of $Y^2$) and has finitely many classes,
			\item
			$E\restr_Y$ is type-definable and has exactly $2^{\aleph_0}$ classes,
			\item
			$E\restr_Y$ is not type-definable and not smooth. In this case, if $E\restr_Y$ is analytic, then $E\restr_Y$ has exactly $2^{\aleph_0}$ classes.
		\end{enumerate}
	\end{cor}
	\begin{proof}
		By Theorem~\ref{thm:main_aut}, we can apply Lemma~\ref{lem:abstract_trich}, which (by Fact~\ref{fct: Borel in various senses}) completes the proof.
	\end{proof}
	
	The following corollary of Theorem~\ref{thm:main_aut} gives a partial answer to the question of possible Borel cardinalities of Lascar strong types (and strong types in general), as raised in \cite{KPS13} (for example, it implies that in NIP theories, the Borel cardinality of $[a]_{\equiv_\KP}/{\equiv_\Lasc}$ is the Borel cardinality of the quotient of a compact Polish group by an $F_\sigma$ subgroup).
	\begin{cor}
		Suppose $E$ is a strong type, while $Y$ is a type-definable and $E$-saturated set such that $\Aut(\fC/\{Y\})$ acts transitively on $Y$ (e.g.\ $E$ refines $\equiv_\KP$ and $Y=[a]_{\equiv_\KP}$ for some $a$). Suppose in addition that $Y$ has NIP. Then $E\restr_Y$ is Borel equivalent to the quotient of a compact Polish group by a subgroup (which is $F_\sigma$, Borel or analytic, respectively, whenever $E$ is such).
	\end{cor}
	\begin{proof}
		This is immediate by Theorem~\ref{thm:main_aut}.
	\end{proof}
	
	\begin{prop}
		\label{prop:closure_has_transitive_action}
		If $E$ is a strong type on a ($\emptyset$-)type-definable set $X$, then for any $a_0\in X$, the set $Y_{a_0}$ of $a\in X$ such that $[a]_E\in \overline{\{[a_0]_E\}}\subseteq X/E$ is type-definable and $E$-saturated. Moreover, $\Aut(\fC/\{Y_{a_0}\})$ is equal to $\{\sigma\in \Aut(\fC)\mid \sigma(a_0)\in Y_{a_0} \}$, and it acts transitively on $Y_{a_0}$.
	\end{prop}
	\begin{proof}
		We may assume without loss of generality that $X=p(\fC)=[a_0]_{\equiv}$ (since $E$ is a strong type, we have $[a_0]_E\subseteq [a_0]_{\equiv}$, which implies that $Y_{a_0}\subseteq [a_0]_{\equiv}$).
		
		Type-definability of $Y_{a_0}$ is straightforward by the definition of the logic topology, $E$-invariance is trivial.
		
		Since $Y_{a_0}\subseteq [a_0]_{\equiv}$, we have that for each $a\in Y_{a_0}$, there is some $\sigma\in \Aut(\fC)$ such that $\sigma(a_0)=a$. It is enough to show that $\sigma\in \Aut(\fC/\{Y_{a_0}\})$. But since $\sigma$ is an automorphism, it acts on $X/E$ by homeomorphisms, so $\sigma(\overline{\{[a_0]_E\}})=\overline{\{[a]_E\}}$. Since $[a]_E\in \overline{\{[a_0]_E\}}$, it follows that $\overline{\{[a]_E\}}\subseteq \overline{\{[a_0]_E\}}$.
		It follows that $\sigma[Y_{a_0}/E]\subseteq Y_{a_0}/E$, so $\sigma[Y_{a_0}]\subseteq Y_{a_0}$.
		
		
		On the other hand, by Remark~\ref{rem:wglike_closedness_to_typdef}, $X/E$ is an $R_0$ space, whence $[a_0]_E\in \overline{\{[a]_E\}}$, so $a_0\in Y_a$. Arguing as in the preceding paragraph, we conclude that $Y_{a_0}=\sigma^{-1}[Y_a]\subseteq Y_a=\sigma[Y_{a_0}]$, so $\sigma[Y_{a_0}]=Y_{a_0}$.
	\end{proof}
	
	The following corollary is the last part of Main~Theorem~\ref{mainthm:smt}. Essentially, it is a generalisation of the main results of \cite{KMS14} and \cite{KM14}/\cite{KR16} (Facts~\ref{fct:KMS_theorem} and \ref{fct:mainA}). It appeared as \cite[Theorem 4.1]{KPR15} (in the paper joint with Krzysztof Krupiński and Anand Pillay). Loosely speaking, it can be seen as a strengthening of Theorem~\ref{thm:nwg} in the ``countable language case'' (as in that case, if $E$ is Borel, then by Fact~\ref{fct:silver}, non-smoothness implies having $2^{\aleph_0}$ classes).
	\begin{cor}
		\label{cor:smt_type}
		Suppose $T$ is countable. If $E$ is a strong type defined on $X=p(\fC)$ for some $p\in S(\emptyset)$ (in countably many variables) and $Y\subseteq X$ is nonempty, type-definable and $E$-saturated, then either $E$ is type-definable, or $E\restr_Y$ is non-smooth.
	\end{cor}
	\begin{proof}
		Suppose $E\restr_Y$ is smooth. We need to show that $E$ is type-definable, which by Proposition~\ref{prop:type-definability_of_relations} is equivalent to $E$ having a type-definable class. Since $E\restr_{Y}$ is smooth, it follows that for every $a\in Y$, $E\restr_{Y_a}$ is also smooth, where $Y_a$ is as in Proposition~\ref{prop:closure_has_transitive_action}. Clearly, it is enough to show that $E\restr_{Y_a}$ has a type-definable class, and thus we may assume without loss of generality that $Y=Y_a$. But then by Proposition~\ref{prop:closure_has_transitive_action}, $\Aut(\fC/\{Y\})$ acts transitively on $Y$, so by Corollary~\ref{cor:trich_plus}, $E\restr_{Y}$ is type-definable, and by Remark~\ref{rem:tdf_iff_restr}, $E$ is type-definable as well.
	\end{proof}
	(See also Corollary~\ref{cor:smt_aut} for a further generalisation (to sets larger than $p(\fC)$).)
	
	Theorem~\ref{thm:main_aut} shows that every quotient of a sufficiently symmetric type-definable set $Y$ by a strong type $E$ is essentially the quotient of a compact Polish group by a subgroup. It is not hard to see that the compact group does not depend on $E$, only on $Y$.
	
	We can actually show more: essentially, we can find one $\hat G$ witnessing Theorem~\ref{thm:main_aut} for all $Y=p(\fC)$, where $p\in S(\emptyset)$, but first, we need an additional lemma.
	
	\begin{lem}
		\label{lem:every_stype_on_m}
		Suppose $T$ is countable.
		Assume that $E$ is a strong type defined on $p(\fC)$ for $p=\tp(a/\emptyset)$ for some countable tuple $a$, while $M$ is an arbitrary countable model, enumerated by $m$.
		
		Then there is a strong type $E'$ on $[m]_{\equiv}$ such that:
		\begin{itemize}
			\item
			$E$ is type-definable [resp. Borel, or analytic, or $F_\sigma$, or relatively definable] if and only if $E'$ is,
			\item
			there are Borel maps $r_1\colon S_m(M)\to S_a(M)$ and $r_2 \colon S_a(M)\to S_m(M)$ such that $r_1$ and $r_2$ are Borel reductions between $(E')^M$ and $E^M$ (in particular, $E'\sim_B E$), satisfying $r_1(\tp(m/M))=\tp(a/M)$ and $r_2(\tp(a/M))=\tp(m/M)$, and
			\item
			the induced maps $r_1'\colon [m]_{\equiv}/E' \to p(\fC)/E$ and $r_2'\colon p(\fC)/E \to [m]_{\equiv}/E'$ are $\Gal(T)$-equivariant homeomorphisms, and $r_2'$ is the inverse of $r_1'$.
		\end{itemize}
		The maps $r_1'$ and $r_2'$ are uniquely determined by $r_1'([\sigma(m)]_{E'})=[\sigma(a)]_{E}$ and $r_2'([\sigma(a)]_{E})=[\sigma(m)]_{E'}$ for all $\sigma \in \Aut(\fC)$.
	\end{lem}
	\begin{proof}
		Let $N\succeq M$ be a countable model containing $a$, and enumerate it by $n\supseteq am$.
		
		Then we have the restriction maps $S_n(N)\to S_a(M)$, $S_n(N)\to S_m(M)$, which fit in the commutative diagram:
		\begin{center}
			\begin{tikzcd}
			S_m(M) \ar[dr, two heads] & S_n(N) \ar[l, two heads] \ar[r, two heads] \ar[d, two heads] & S_a(M) \ar[d, two heads] \\
			&\Gal(T) \ar[r, two heads] & p(\fC)/E.
			\end{tikzcd}
		\end{center}
		In this diagram, the maps to $\Gal(T)$ are given by Fact~\ref{fct:sm_to_gal}, while the map $\Gal(T)\to p(\fC)/E$ is the orbit map $\sigma\Autf(\fC)\mapsto [\sigma(a)]_E$ (cf. Proposition~\ref{prop:gal_action}).
		
		Recall from Fact~\ref{fct: Borel in various senses} that $E$ is type-definable [resp. Borel, analytic, $F_\sigma$, relatively definable] if and only if the induced relation $E^M$ on $S_a(M)$ is closed [resp. Borel, analytic, $F_\sigma$, clopen]. Note also that $E^M=E|_{S_{a}(M)}$ (using Definition~\ref{dfn:induced_relation}).
		
		Note that $E|_{S_m(M)}$ and $E|_{S_n(N)}$ are both induced by the same left invariant equivalence relation $E|_{\Gal(T)}$ on $\Gal(T)$ (left invariance holds because the map $\Gal(T)\to p(\fC)/E$ is left $\Gal(T)$-equivariant, as the orbit map of a left action).
		
		
		Let $E'$ be the $\Aut(\fC/M)$-invariant equivalence relation on $[m]_{\equiv}$ such that $(E')^M$ is $E|_{S_m(M)}$. It is $\Aut(\fC)$-invariant by construction (e.g.\ because $E|_{\Gal(T)}$ is left invariant), and it is clearly bounded by the size of $p(\fC)/E$. We will show that it satisfies the conclusion.
		
		Since $E|_{S_n(N)}$ is the pullback of both $E|_{S_m(M)}$ and $E|_{S_a(M)}$ (and so, a preimage of each by a continuous surjection), the first part follows by Proposition~\ref{prop:preservation_properties}.
		
		Fact~\ref{fct:borel_section} gives us Borel sections $S_m(M)\to S_n(N)$ and $S_a(M)\to S_n(N)$ of the restriction maps, and we can assume without loss of generality that each section maps $\tp(m/M)$ or $\tp(a/M)$ (respectively) to $\tp(n/N)$ (possibly by changing the value of the section at one point). Those sections, composed with the appropriate restrictions from $S_n(N)$, yield Borel maps $r_1\colon S_m(M)\to S_a(M)$ and $r_2\colon S_a(M)\to S_m(M)$, which (by the last sentence) take $\tp(m/M)$ to $\tp(a/M)$ and vice versa. These maps are clearly Borel reductions between $E|_{S_m(M)}$ and $E^M$ (passing via $E|_{S_n(N)}$). Denote by $r_1',r_2'$ the induced maps between the class spaces (as in the statement of the lemma), where we freely identify various homeomorphic quotient spaces (e.g.\ $p(\fC)/E$ and $S_a(M)/E^M$).
		
		Now, note that given any $\sigma\in \Aut(\fC)$, the restriction of $\tp(\sigma(n)/N)\in S_n(N)$ to $S_m(M)$ is $\tp(\sigma(m)/M)$ and likewise, the restriction to $S_a(M)$ is $\tp(\sigma(a)/M)$. It follows easily that for every $\sigma\in \Aut(\fC)$, we have $r_1'([\tp(\sigma(m)/M]_{E|_{S_m(M)}})=[\tp(\sigma(a)/M)]_{E^M}$ and likewise, $r_2'([\tp(\sigma(a)/M)]_{E^M})=[\tp(\sigma(m)/M)]_{E|_{S_m(M)}}$. In particular, $r_1'$ and $r_2'$ are bijections with $r_2'$ being the inverse of $r_1'$, and they are $\Gal(T)$-equivariant.
		
		Note that all the maps in the diagram are quotient maps, so in particular, the composed map $S_m(M)\to p(\fC)/E$ is a quotient map. It is easy to see that this map is the composition of the bijection $r_1'$ and the quotient map $S_m(M)\to [m]_{\equiv}/E'$, which implies that $r_1'$ is a homeomorphism, and hence, so is $r_2'$.
		
		Finally note that the conditions $r_1(\tp(m/M))=\tp(a/M)$ and $r_2(\tp(a/M))=\tp(m/M)$, together with $\Gal(T)$-equivariance of $r_1'$ and $r_2'$, imply that $r_1'$ and $r_2'$ are determined by $r_1'([\sigma(m)]_{E'})=[\sigma(a)]_{E}$ and $r_2'([\sigma(a)]_{E})=[\sigma(m)]_{E'}$, for all $\sigma \in \Aut(\fC)$.
	\end{proof}
	The next theorem (along with Theorem~\ref{thm:main_aut}) completes Main~Theorem~\ref{mainthm_group_types}. It is \cite[Theorem 7.13]{KR18}, and is the main result of that paper (joint with Krzysztof Krupiński).
	\begin{thm}
		\label{thm:main_galois}
		Let $T$ be an arbitrary countable theory, and let $M$ be any countable ambitious model of $T$ (such a model exists by Proposition~\ref{prop:amb_exist}).
		
		Consider the ambit $(\Aut(M),S_m(M),\tp(m/M))$, and let $\hat G$ be the compact Polish group $u\cM/H(u\cM)\Core(D)$ (as in Corollary~\ref{cor:Polish_quotient_Core(D)}). Then the orbit map $E(\Aut(M),S_m(M))\to S_m(M)$, $f\mapsto f(\tp(m/M))$ induces a topological group quotient mapping $\hat r\colon \hat G\to \Gal(T)$ (identified with $S_m(M)/{\equiv_\Lasc^M}$ via Fact~\ref{fct:sm_to_gal}) with the following property.
		
		Suppose $E$ is a strong type defined on $p(\fC)$ for some $p\in S(\emptyset)$ (in countably many variables). Fix any $a\models p$.
		
		Denote by $r_{[a]_E}$ the orbit map $\Gal(T)\to p(\fC)/E$ given by $\sigma \Autf(\fC)\mapsto[\sigma(a)]_E$ (i.e. the orbit map of the natural action of $\Gal(T)$ on $p(\fC)/E$ from Proposition~\ref{prop:gal_action}).
		
		Then for $\hat r_{[a]_E}:=r_{[a]_E}\circ \hat r$ and $H=\ker \hat r_{[a]_E}:=\hat r_{[a]_E}^{-1}[[a]_E]$ we have that:
		\begin{enumerate}
			\item
			$H\leq \hat G$ and the fibres of $\hat r_{[a]_E}$ are the left cosets of $H$,
			\item
			$\hat r_{[a]_E}$ is a topological quotient map, and so $p(\fC)/E$ is homeomorphic to $\hat G/H$,
			\item
			$E$ is type-definable if and only if $H$ is closed,
			\item
			$E$ is relatively definable on $p(\fC) \times p(\fC)$ if and only if $H$ is clopen,
			\item
			if $E$ is Borel [resp. analytic, or $F_\sigma$], then $H$ is Borel [resp. analytic, or $F_\sigma$],
			\item
			$E_H\leq_B E$, where $E_H$ is the relation of lying in the same left coset of $H$.
		\end{enumerate}
		
		Moreover, if $T$ has NIP (or, more generally, if $M$ is a tame ambitious model --- cf.\ Definition~\ref{dfn:tame_model}), then $E_H\sim_B E$.
	\end{thm}
	\begin{proof}
		By Lemma~\ref{lem:lascar_grouplike}, $\equiv_\Lasc^M$ is uniformly properly group-like (in the ambit $(\Aut(M),S_m(M),\tp(m/M))$), so by Theorem~\ref{thm:main_abstract}, we obtain the epimorphism $\hat r \colon \hat G\to [m]_{\equiv}/{\equiv_\Lasc}$ (identified with $\Gal(T)$ and $S_m(M)/{\equiv_\Lasc}$ via Fact~\ref{fct:sm_to_gal} and Fact~\ref{fct:logic_by_type_space}).
		
		Now, fix an $p$, $a\models p$ and $E$. Note that Lemma~\ref{lem:every_stype_on_m} yields a strong type $E'$ on $[m]_\equiv$ and a map $r_1'\colon [m]_\equiv/E' \to p(\fC)/E$ satisfying all the conclusions of that lemma, in particular, $r_1'([\sigma(m)]_{E'})= [\sigma(a)]_E$ for any $\sigma \in \Aut(\fC)$. Therefore, $r_{[a]_E}=r_1'\circ r_{[m]_{E'}}$, and so $\hat r_{[a]_E}=r_1'\circ \hat r_{[m]_{E'}}$. This, together with the conclusions of Lemma~\ref{lem:every_stype_on_m}, shows that we can assume without loss of generality that that $m=a$.
		
		Note that the Borel cardinality of $E$ is by definition the Borel cardinality of $E^M$, $p(\fC)/E$ is homeomorphic to $S_m(M)/E^M$, and by Fact~\ref{fct: Borel in various senses}, we can translate topological and descriptive properties of $E$ to those of $E^M$. Recall also that by definition, if $M$ is a tame ambitious model (which is true for any ambitious model under NIP, see Corollary~\ref{cor:NIP_implies_tame}), then the dynamical system $(\Aut(M),S_m(M))$ is tame.
		
		Now, if $E$ is a strong type defined on $[m]_{\equiv}$, then it is refined by $\equiv_\Lasc$. Therefore, $E^M$ is refined by $\equiv_\Lasc^M$; of course, $\Gal(T)=S_m(M)/{\equiv_\Lasc^M}$ acts on itself preserving $E|_{\Gal(T)}$ so $E^M$ is dominated by $\equiv_\Lasc^M$, and as such it is weakly uniformly properly group-like in the ambit $(\Aut(M),S_m(M),\tp(m/M))$. Thus Theorem~\ref{thm:main_abstract} applies to $E^M$, and the function in its conclusion is just $\hat r_{[m]_{E}}\colon \hat G\to [m]_{\equiv}/E=S_m(M)/E^M$. By the preceding paragraph, the properties of $\hat r_{[m]_{E}}$ given by Theorem~\ref{thm:main_abstract} give us the properties postulated by this theorem, which completes the proof.
	\end{proof}
	
	
	
	\begin{cor}
		In Theorem~\ref{thm:main_galois}, we also have $E_H\sim_B E$ if $E$ is coarser than $\equiv_\KP$.
	\end{cor}
	\begin{proof}
		If $E$ is coarser than $\equiv_\KP$, then by Theorem~\ref{thm:main_over_KP}, the orbit map $r_{[a]_E,KP}$ from $\Gal_\KP(T)$ into $p(\fC)/E$ gives us $E\sim_B E|_{\Gal_\KP(T)}$
		
		On the other hand, if we denote by $\hat r_{KP}$ the function $\hat G\to S_m(M)/{\equiv_\KP^M}=\Gal_\KP(T)$ which we get by applying Theorem~\ref{thm:main_galois} to $\equiv_\KP$ on $[m]_{\equiv}$, then $\hat r_{KP}$ is a continuous epimorphism of compact Polish groups. Because $\hat r_{[a]_E}=r_{[a]_E,KP}\circ \hat r_{KP}$, it follows by Fact~\ref{fct:borel_section} that $E|_{\hat G}\sim_B E|_{\Gal_\KP(T)}\sim_B E$.
		
		(Alternatively, this follows from Remark~\ref{rem:strengthening}.)
	\end{proof}
	
	\begin{rem}
		In Theorem~\ref{thm:main_galois}, it seems plausible that we may also have $E_H\sim_B E$ for all $E$ on a tame $p(\fC)$. If $p$ is realised in the ambitious model $M$ chosen in the proof, then it follows from Corollary~\ref{cor:tame_dominated}, but in general (e.g.\ if $a$ enumerates a larger model), there seems to be no obvious argument. The main obstacle is that there is no clear connection between the Ellis groups of flows $(\Aut(M),S_m(M))$ as we vary the ambitious model $M$. It is possible that one can use methods similar to those introduced in \cite{KNS17} to show this stronger result.\xqed{\lozenge}
	\end{rem}
	
	\begin{rem}
		Theorem~\ref{thm:main_galois} can be extended in the following way: given a subgroup $G_0\leq \Aut(\fC)$ containing $\Autf(\fC)$, and such that $G_0/\Autf(\fC)$ is closed in $\Gal(T)$, we can find a group $\hat G_0$ which witnesses Theorem~\ref{thm:main_aut} for all sets of the form $Y_a:=G_0\cdot a$ (for some countable tuple $a$) in such a way that the action of $\hat G_0$ on $Y/E$ factors through $G_0/\Autf(\fC)$. The proof is analogous, except we have to choose a countable model ambitious relative to $G_0/{\Autf(\fC)}$, and prove a variant of Lemma~\ref{lem:every_stype_on_m} for $Y_a$ in place of $p(\fC)$.
	\end{rem}
	
	
	
	\section{Actions of type-definable groups}
	In this section, we consider a group $G$ acting on a set $X$, such that $G,X$ and the action are type-definable (over $\emptyset$, unless specified otherwise).
	
	We also consider, for a small model $M$, the group $G(M)$ acting on $S_G(M)=G_M$ (the space of types over $M$ concentrated on $G$) and on $X_M$ (the space of $M$-types of elements of $X$). Note that for every $g\in G$, the map $x\mapsto gx$ is type-definable over $g$. It follows immediately that $G(M)$ acts on $X_M$ by homeomorphisms.
	
	\index{EG000@$E_{G^{000}_\emptyset}$}
	Throughout the section, we denote by $E_{G^{000}_\emptyset}$ the coset equivalence relation of $G^{000}_\emptyset$ on $G$.
	
	\begin{rem}
		Note that if $G$ is a group invariant over a small model $M$, then it is easy to see that the set of elements of $G$ invariant over $M$ is a group.
		
		On the other hand, if $g\in G$ is invariant over $M$, then every coordinate of $g$ is definable over $M$ (because it definable and fixed by $\Aut(\fC/M)$), and as such, it is an element of $G$. It follows that $g$ is a tuple of elements of $G$, so $g\in G(M)$, so $G(M)$ is always a subgroup of $G$.\xqed{\lozenge}
	\end{rem}
	
	
	\subsection*{Lemmas}
	We intend to apply results of Chapter~\ref{chap:grouplike}. In the following lemmas, we will show that their hypotheses are satisfied in the case of transitive type-definable group actions.
	
	\begin{lem}
		\label{lem:quot_by_normal_grouplike}
		Suppose $G$ is a $\emptyset$-type-definable group and $N\unlhd G$ is an invariant normal subgroup of bounded index. Suppose $M$ is a model such that $G(M)=G(M)\cdot \tp(e/M)$ is dense in $S_G(M)$.
		
		Then for the relation $E_N$ of lying in the same coset of $N$, $E_N^M$ is group-like on $(G(M),S_G(M))$.
	\end{lem}
	\begin{proof}
		Note that $G/E_N=G/N$ and so, by Fact~\ref{fct:logic_by_type_space}, topologically $S_G(M)/E_N^M=G/N$. On the other hand, $G/N$ is a topological group by Fact~\ref{fct:quotient_by_bounded_subgroup}. The fact that $G(M)\to G/N$ is a group homomorphism is trivial.
	\end{proof}
	
	
	\begin{lem}[pseudocompleteness for type-definable groups]
		\label{lem:pseudocompleteness_for_groups}
		Let $M$ be a model, and suppose $G$ is a group type-definable over $M$. Let $N\succeq M$ be $\kappa^+$-saturated, where $\kappa$ is the maximum of $\lvert M\rvert$ and the length of an element of $G$, considered as a tuple (e.g.\ $N=\fC$ and $M$ is small in $\fC$).
		
		Then whenever $(g_i)_i$ and $(p_i)_i$ are nets in $G(M)$ and $S_G(M)$ (respectively) such that $\tp(g_i/M)\to q_1$, $p_i\to q_2$ and $g_i(p_i)\to q_3$ for some $q_1,q_2,q_3\in S_G(M)$, there are $g'_1,g'_2\in G(N)$ such that $\tp(g'_1/M)=q_1$, $\tp(g'_2/M)=q_2$ and $\tp(g'_1g'_2/M)=q_3$. (This is pseudocompleteness for $\tilde G=G(N)$, $X=S_G(M)$ and the map $\tilde G\to X$ given by $g\mapsto \tp(g/M)$, see Definition~\ref{dfn:prop_glike}.)
	\end{lem}
	\begin{proof}
		Take any net $(a_i)_i$ in $N$ such that for all $i$, $a_i\models p_i$. Then for each $\varphi_1,\varphi_2,\varphi_3$ in $q_1,q_2,q_3$ (respectively), we have for sufficiently large $i$ that $N\models \varphi_1(g_i)\land \varphi_2(a_i)\land \varphi_3(g_ia_i)$. The conclusion follows easily by compactness.
	\end{proof}
	
	\begin{lem}
		If $M$ is any model, and If $G$ is an $M$-type-definable group, then the set $F_0=\{\tp(g_1^{-1}g_2/M)\mid g_1\equiv_M g_2\in G \}$ is closed in $S_G(M)$.
	\end{lem}
	\begin{proof}
		Straightforward.
	\end{proof}
	
	When reading the proof of Lemma~\ref{lem:coset_rel_is_glike}, it may be helpful to compare the diagram below to the one in Definition~\ref{dfn:prop_glike}.
	
	\begin{center}
		\begin{tikzcd}
			G(M)\ar[r]\ar[dr] & E(G(M),S_G(M))\ar[d,"R"]\ar[dr,"r"] \\
			G(\fC)\ar[r] & S_G(M)\ar[r] & G/G^{000}_{\emptyset}
		\end{tikzcd}
	\end{center}
	\begin{lem}
		\label{lem:coset_rel_is_glike}
		Given a type-definable group $G$ and a small model $M$ such that $G(M)\cdot \tp(m/M)$ is dense in $S_G(M)$, we have that the coset relation $E_{G^{000}_{\emptyset}}^M$ on $S_G(M)$ is uniformly properly group-like (according to Definition~\ref{dfn:unif_prop_glike}).
	\end{lem}
	\begin{proof}
		From the preceding lemmas it follows easily that $\tilde G=G=G(\fC)$ witnesses that $E_{G^{000}_{\emptyset}}^M$ is properly group-like, with $[\tilde g]_{\equiv}=\tp(\tilde g/M)$ (in the sense of Definition~\ref{dfn:prop_glike}).
		
		To see that it is uniformly properly group-like, denote by $A$ the symmetric subset of $G$ consisting of products $g^{-1}g'$, where $g\equiv_M g'$, and let $\mathcal E$ be the family of sets of the form $F_n=\{
		(\tp(g/M),\tp(g'g/M))\mid g \in G\land g'\in A^n \}$ (where $A^n$ is the set of all products of $n$ elements of $A$). They are clearly closed, symmetric and contain the diagonal in $S_G(M)^2$.
		
		We have that $F_{2n+1}\supseteq F_{n}\circ F_n$. Indeed, suppose we have two pairs in $F_n$: $(\tp(g/M),\tp(g'g/M))$ and $(\tp(h/M), \tp(h'h/M))$ (i.e.\ $g',h'\in A^n$) and $\tp(h/M)=\tp(g'g/M)$, then for $h''=g'g$ we have $h\equiv_M h''$, so $h(h'')^{-1}\in A$, and
		\[
			h'h=h'(h(h'')^{-1})h''=h'(h(h'')^{-1})g'g,
		\]
		so we have $(\tp(g/M),\tp(h'h/M))\in F_{2n+1}$.
		
		On the other hand, suppose $(\tp(e/M),\tp(g/M))\in F_n$. Then for some $g'\in A^n$ we have $g\equiv_M g'$. Since $A^n$ is clearly invariant over $M$, it follows that $g\in A^n$, so for any $g''\in G$ we have $(\tp(g''/M),\tp(g''g/M))\in F_n\subseteq F_{2n+1}$, which completes the proof.
	\end{proof}
	
	\begin{prop}
		\label{prop:bdd_iff_invariant}
		Suppose $A\subseteq \fC$ is a small set, $G$ is a group acting transitively on a set $X$, and $E$ is a $G$-invariant equivalence relation on $X$. Suppose in addition that $G$, $X$, $E$ and the action are all $\Aut(\fC/A)$-invariant.
		
		Then $E$ is bounded if and only if its classes are setwise $G^{000}_A$-invariant.
	\end{prop}
	\begin{proof}
		Note that since $E$ is $G$-invariant, $G$ acts on $X/E$.
		
		If $E$ is bounded, then $X/E$ is small, so the kernel of this action has small index. By the assumptions, the kernel is also invariant over $A$, so it contains $G^{000}_A$, which implies that the classes are $G^{000}_A$-invariant.
		
		In the other direction, if all classes of $E$ are setwise $G^{000}_A$-invariant, then for any $x_0\in X$, the assignment $gG^{000}_A\mapsto [g\cdot x_0]_E$ yields a well-defined function $G/G^{000}_A\to X/E$. Because $G$ acts transitively on $X$, this function is surjective. In particular, $\lvert X/E\rvert\leq [G:G^{000}_A]$, so $E$ is bounded.
	\end{proof}
	
	
	
	\begin{lem}
		\label{lem:weakly_grouplike_tdf}
		Suppose $G$ is a type-definable group acting type-definably and transitively on a type-definable set $X$ (all without parameters).
		
		Suppose in addition that $E$ is a $G$-invariant bounded invariant equivalence relation on $X$.
		
		Let $M$ be a small model such that $X(M)$ is nonempty, while $G(M)$ is dense in $S_G(M)$, and choose some $x_0\in X(M)$.
		
		Then the relation $E_{G^{000}_\emptyset}^M$ on the ambit $(S_G(M),\tp(e/M))$ dominates $E^M$ on the ambit $(X_M,\tp(x_0/M))$, and in particular (by Lemma~\ref{lem:coset_rel_is_glike}), $E^M$ is weakly uniformly properly group-like on the ambit $(G(M),X_M,\tp(x_0/M))$.
	\end{lem}
	\begin{proof}
		Since $G$ acts type-definably on $X$, it follows that the orbit map $g\mapsto g\cdot x_0$ is type-definable over $M$, so it induces a continuous map $S_G(M)\to X_M$. Because the action of $G$ on $X$ is transitive, this map is onto. By Proposition~\ref{prop:bdd_iff_invariant}, $E$-classes are setwise $G^{000}_\emptyset$-invariant, which implies that $E^M|_{S_G(M)}$ (see Definition~\ref{dfn:induced_relation}) is refined by $E_{G^{000}_\emptyset}^M$. Since $E$ is $G$-invariant, it follows that $S_G(M)/E_{G^{000}_\emptyset}^M=G/G^{000}_\emptyset$ acts on $X/E=X_M/E^M$, witnessing the domination.
	\end{proof}
	
	\begin{prop}
		\label{prop:amb_for_groups}
		Let $x_0\in \fC$ be an arbitrary tuple, and let $G$ be a $\emptyset$-type-definable group, consisting of tuples of length at most $\lambda$.
		
		Then there is a model $M$ of cardinality at most $\lvert T\rvert+\lvert x_0\rvert+\lambda$ containing $x_0$, such that $G(M)$ is dense in $S_G(M)$.
	\end{prop}
	\begin{proof}
		The proof is analogous to that of Proposition~\ref{prop:amb_exist}
		
		Roughly, start with any model $M_0$ containing $x_0$, and find a small group $G_0\leq G$ such that $\{\tp(g_0/M_0)\mid g_0\in G_0 \}$ is dense in $S_G(M_0)$, and expand $M_0$ to a small model $M_1$ containing $G_0$, and continue. After $\omega$ steps, take the union of the resulting elementary chain.
		
		(Note that if $G$ is a \emph{definable} group, then $G(M)$ is always dense in $S_G(M)$.)
	\end{proof}
	
	We have the following proposition, analogous to Corollary~\ref{cor:NIP_implies_tame}.
	\begin{prop}
		\label{prop:nip_tame_group}
		Suppose $G$ is a type-definable group acting type-definably on a type-definable set $X$. Let $M$ be a model over which $G,X$ and the action are type-definable. Then if $X$ has NIP, then the dynamical system $(G(M),X_M)$ is tame (cf.\ Definition~\ref{dfn:NIP_set} and Definition~\ref{dfn:tame_function_system}).
	\end{prop}
	\begin{proof}
		The proof is by contraposition. Suppose $(G(M),X_M)$ is untame. We will show that $X$ has IP (i.e.\ does not have NIP).
		
		Since $X_M$ is totally disconnected, by Proposition~\ref{prop:dyn_BFT}, there is a clopen subset $U\subseteq X_M$ and a sequence $(g_n)_{n\in \bN}$ in $G(M)$ such that the sets $g_nU$ are independent. Fix a formula $\varphi(x)$ with parameters in $M$ giving $U$ (i.e.\ such that $[\varphi(x)]\cap X_M=U$).
		
		Write $\mu$ for the multiplication $G\times X\to X$. Note that since $\mu$ is type-definable, the preimage $\mu^{-1}[\varphi(\fC)]$ is relatively definable over $M$ in $G\times X$ (because it is type-definable and so is its preimage, $\mu^{-1}[\neg \varphi(\fC)]$). Let $\psi(y,x)$ be a formula (with parameters from $M$) such that $\psi(G,X)=\mu^{-1}[\varphi(\fC)\cap X]$. Then by the assumption, $\psi(g_n,\fC)\cap X=g_n(\varphi(\fC)\cap X)$ are independent subsets of $X$, so $\psi$ does not have NIP on $G\times X$, so $X$ does not have NIP.
	\end{proof}
	
	The following proposition shows that NIP assumption on $G$ implies NIP for all transitive (type-definable) $G$-spaces.
	\begin{prop}
		The image of an NIP set by a type-definable surjection is NIP. In particular, if $G$ is a type-definable group with NIP, acting transitively and type-definably on $X$, then $X$ also has NIP.
	\end{prop}
	\begin{proof}
		The proof is by contraposition. We will show that if the range of a type-definable function does not have NIP, then neither does the domain.
		
		Fix a small set of parameters $A$, a set $X$, and a surjection $f\colon Z\to X$, all type-definable over $A$.
		
		Suppose $X$ has IP. Then there is a formula $\varphi(x,y)$ witnessing it; we may assume without loss of generality that all parameters in $\varphi$ are from $A$ (making it larger if necessary). In particular, by Remark~\ref{rem:NIP_indiscernible}, we can find a sequence $(b_n)_{n\in\bN}$, indiscernible over $A$, such that the sets $\varphi(\fC,b_n)\cap X$ are independent in $X$. Then clearly the sets $f^{-1}[\varphi(\fC,b_n)\cap X]$ are independent in $Z$, and we only need to show that they are uniformly definable.
		
		First, note that the set $f^{-1}[\varphi(\fC,b_0)]$ is relatively definable in $Z$ (it is obviously type-definable, and the same is true about its complement in $Z$), so there is some definable set $W$ such that $W\cap Z=f^{-1}[\varphi(\fC,b_0)]$. Now, since the sequence $(b_n)_{n\in \bN}$ is indiscernible over $A$, we can find, for each $n$, some automorphism $\sigma_n\in \Aut(\fC/A)$ such that $\sigma_n(b_0)=b_n$. But since $f$ and $X$ are invariant over $A$ and $\varphi(x,y)$ has parameters only from $A$, it follows that $\sigma_n(W)\cap Z=f^{-1}[\varphi(\fC,\sigma_n(b_0))\cap X]=f^{-1}[\varphi(\fC,b_n)\cap X]$. Since $\sigma_n(W)$ are clearly uniformly definable, we are done.
	\end{proof}
	\begin{rem}
		Note also that it is not hard to see that if $(G(M),G_M)$ is a tame dynamical system, then for every $M$-type-definable transitive $G$-space $X$ with nonempty $X(M)$, the system $(G(M),X_M)$ is also tame: under those hypotheses, we can have a $G(M)$-ambit morphism $(G_M,\tp(e/M))\to (X_M,\tp(x_0/M))$ (where $x_0\in X(M)$ is arbitrary), and apply Fact~\ref{fct:tame_preserved}.\xqed{\lozenge}
	\end{rem}
	
	\subsection*{Results for type-definable group actions}
	
	Now, Lemma~\ref{lem:weakly_grouplike_tdf} allows us to apply preceding results, including Theorem~\ref{thm:general_cardinality_intransitive}, Theorem~\ref{thm:general_cardinality_transitive} and Theorem~\ref{thm:main_abstract}. In particular, we have the following theorem (originally, \cite[Theorem 8.4]{KR18}, joint with Krzysztof Krupiński).
	
	
	
	\begin{thm}
		\label{thm:main_tdf}
		Suppose that the theory is countable, and $A\subseteq \fC$ is a countable set of parameters.
		
		Let $G$ be an type-definable group (of countable tuples), acting type-definably and transitively on a type-definable set $X$ (of countable tuples), all with parameters in $A$. Let $E$ be a bounded, $G$-invariant and $\Aut(\fC/A)$-invariant equivalence relation on $X$.
		
		Then there is a compact Polish group $\hat G$ acting continuously on $X/E$, and such that for any $x_0\in X$, the stabiliser $H$ of $[x_0]_E$, and the orbit map $\hat r\colon \hat G\to X/E$, $\hat g\mapsto \hat g\cdot [x_0]_E$, have the following properties:
		\begin{enumerate}
			\item
			$H\leq \hat G$ and fibres of $\hat r$ are exactly the left cosets of $H$ (so $\hat G/E|_{\hat G}=\hat G/H$),
			\item
			$\hat r$ is a topological quotient map (so it induces a homeomorphism of $\hat G/H$ and $X/E$),
			\item
			$E$ is relatively definable (as a subset of $X^2$) or type-definable if and only if $H$ is clopen or closed (respectively)
			\item
			if $E$ is $F_\sigma$, Borel, or analytic (respectively), then so is $H$,
			\item
			$\hat G/H\leq_B E$.
		\end{enumerate}
		Furthermore, if $X$ has NIP (in particular, if $G$ has NIP or, yet more generally, if $T$ has NIP), then $\hat G/H\sim_B E$.
	\end{thm}
	\begin{proof}
		Note first that we may assume without loss of generality that $A=\emptyset$ (if necessary, we may add some countably many parameters to the language).
		
		Fix $x_0$ and find a countable model $M$ as in Proposition~\ref{prop:amb_for_groups}. Then Lemma~\ref{lem:weakly_grouplike_tdf} applies, and we can apply Theorem~\ref{thm:main_abstract}, arguing as in the proof of Theorem~\ref{thm:main_aut}.
		
		More precisely, by Proposition~\ref{prop:amb_for_groups}, we can fix a model $M$ satisfying the hypotheses of Lemma~\ref{lem:weakly_grouplike_tdf}, and then for any $x_0\in X(M)$, $(G(M),X_M,\tp(x_0/M))$ is an ambit and $E^M$ is weakly uniformly properly group-like. Furthermore, by Proposition~\ref{prop:nip_tame_group}, if $X$ has NIP, then $(G(M),X_M)$ is tame.
		
		Recall that we identify $X/E$ and $X_M/E^M$ (and the identification is homeomorphic), the Borel cardinality of $E$ is by definition the Borel cardinality of $E^M$, and by Fact~\ref{fct: Borel in various senses}, we have that $E$ is relatively definable in $X^2$, type-definable, $F_\sigma$, Borel, or analytic if and only if $E^M$ is clopen, closed, $F_\sigma$, Borel or analytic (respectively).
		
		Thus, the by the third paragraph, the assumptions of Theorem~\ref{thm:main_abstract} are satisfied, and by the fourth paragraph, its conclusion gives us the desired $\hat G$, action and $\hat r$.
	\end{proof}
	
	\begin{rem}
		By going back to Theorem~\ref{thm:main_abstract}, we see that the group $\hat G$ in Theorem~\ref{thm:main_tdf} is actually the quotient $u\cM/H(u\cM)\Core(D)$ calculated for the ambit $(G(M),S_X(M),\tp(x_0/M))$.
		\xqed{\lozenge}
	\end{rem}
	
	\begin{rem}
		In Theorem~\ref{thm:main_tdf}, if the stabiliser of $[x_0]_E$ is normal in $G$, then so is the stabiliser in $G/G^{000}_A$, which gives $X/E$ a topological group structure such that the orbit map $G/G^{000}_A\to X/E$ (at $[x_0]_E$) is a homomorphism.
		
		It is not hard to see that this implies that that $E^M$ satisfies the assumptions of Proposition~\ref{prop:wgl_homom}, and so by Theorem~\ref{thm:main_abstract}(7), $H$ is normal.\xqed{\lozenge}
	\end{rem}
	
	
	We can also apply Corollary~\ref{cor:metr_smt_cls}, yielding the following. See also Corollary~\ref{cor:smt_def} for related statement which applies to intransitive actions.
	\begin{cor}
		\label{cor:trich+_tdf}
		Suppose $T$ is countable. Let $A\subseteq \fC$ be countable.
		
		Suppose in addition that $G$ is a type-definable group, and $X$ is an type-definable set of countable tuples on which $G$ acts transitively and type-definably (all with parameters in $A$), while $E$ is a bounded $G$-invariant and $\Aut(\fC/A)$-invariant equivalence relation on $X$. Then exactly one of the following holds:
		\begin{enumerate}
			\item
			$E$ is relatively definable (as a subset of $X^2$) and has finitely many classes,
			\item
			$E$ is type-definable and has exactly $2^{\aleph_0}$ classes,
			\item
			$E$ is not type-definable and not smooth. In this case, if $E$ is analytic, then it has exactly $2^{\aleph_0}$ classes.
		\end{enumerate}
		In particular, $E$ is smooth if and only if it is type-definable.
		
		Furthermore, if $X$ has NIP, then the Borel cardinality of $E$ is the Borel cardinality of the coset equivalence relation of a subgroup of a compact Polish group (which is $F_\sigma$, Borel or analytic, respectively, whenever $E$ is such).
	\end{cor}
	\begin{proof}
		By Theorem~\ref{thm:main_tdf}, we can apply Lemma~\ref{lem:abstract_trich}, which (by Fact~\ref{fct: Borel in various senses}) completes the proof, apart from the ``furthermore'' part, which follows directly from Theorem~\ref{thm:main_tdf}.
	\end{proof}
	
	The following corollary (Main~Theorem~\ref{mainthm:tdgroup}) is a strengthening of Fact~\ref{fct:KM_about_groups} from \cite{KM14}; it partially appeared in \cite{KPR15} and in \cite{KR18} (cf.\ the comments preceding Main~Theorem~\ref{mainthm:tdgroup}).
	\begin{cor}
		\label{cor:trich_tdgroups}
		Suppose $G$ is a type-definable group, while $H\leq G$ is an analytic subgroup, invariant over a small set. Then exactly one of the following holds:
		\begin{itemize}
			\item
			$[G:H]$ is finite and $H$ is relatively definable,
			\item
			$[G:H]\geq 2^{\aleph_0}$, but is bounded, and $H$ is not relatively definable.
			\item
			$[G:H]$ is unbounded (i.e.\ not small).
		\end{itemize}
		In particular, $[G:H]$ cannot be infinite and smaller than $2^{\aleph_0}$.
		
		Moreover, if the language is countable, $G$ consists of countable tuples, and $G$ and $H$ are invariant over a countable set, then we can divide the second case further: either $H$ is type-definable, or $G/H$ is not smooth.
	\end{cor}
	\begin{proof}
		Choose $M$ as in Proposition~\ref{prop:amb_for_groups} for $x_0=e_G$.
		
		If $[G:H]$ is unbounded, there is nothing to prove. Otherwise, the left coset equivalence relation $E_H$ on $G$ is bounded, $G$-invariant, and invariant over the set over which $G$ and $H$ are invariant.
		Furthermore, $G(M)$ is dense in $S_G(M)$, so it has a dense orbit in $G/H$, i.e.\ $G(M)\cdot e_{G/H}=G(M)/H$ is dense in $G/H=S_G(M)/E_H^M$.
		
		Therefore, we can apply Lemma~\ref{lem:weakly_grouplike_tdf} to $E_H$, and it follows that Theorem~\ref{thm:general_cardinality_transitive} applies with $X=Y=S_G(M)$ and $E_H^M$. It is easy to see that $E_H^M$ is clopen if and only if both $E_H$ and $H$ are relatively definable, which completes the proof of the trichotomy.
		
		Under the countability assumptions, we can apply Corollary~\ref{cor:trich+_tdf} with $X=G$ and $E=E_H$, noting that $E_H$ is type-definable if and only if $H$ is type-definable. This gives us the ``moreover'' part.
	\end{proof}
	
	\begin{rem}
		It is possible to have $[G:H]=2$ for an invariant and not relatively definable $H$ (see \cite[Example 3.39]{KM14}), but then $H$ is necessarily non-analytic. (Indeed, in the cited example, the $H$ is only obtained existentially, and is a kind of ``Vitali set".)\xqed{\lozenge}
	\end{rem}
	
	
	\begin{rem}
		In Theorem~\ref{thm:main_tdf}, in the ``Furthermore" part, one can weaken the assumption that $X$ has NIP to say only that there is no $\varphi(x)$ with parameters in $M$ such that $\{g\cdot [\varphi(x)]\mid g\in G(M) \}$ contains an independent family, as that is enough to guarantee that $(G(M),S_G(M))$ is tame.
		\xqed{\lozenge}
	\end{rem}
	
	\begin{rem}
		One may also show that we have the analogue of Lemma~\ref{lem:every_stype_on_m}, and using that, obtain an analogue of Theorem~\ref{thm:main_galois}. Roughly speaking, given a fixed $G$ and $A$, there is a single $\hat G$ witnessing Theorem~\ref{thm:main_tdf} for all $X$ and $E$.\xqed{\lozenge}
	\end{rem}
	
	\section{Other applications in model theory}
	\label{sec:other_apps}
	As mentioned in the introduction, Theorem~\ref{thm:main_abstract} (as well as Theorems~\ref{thm:general_cardinality_intransitive} and \ref{thm:general_cardinality_transitive}) may be used to deduce virtually all the similar results in the model-theoretic contexts, either by directly showing that some equivalence relation is (weakly) uniformly properly group-like, or by some reduction to Theorem~\ref{thm:main_aut} or Theorem~\ref{thm:main_tdf}.
	
	Before, we have seen how we can recover and even improve the results from the papers \cite{Ne03}, \cite{KMS14}, \cite{KM14}, \cite{KP17}, \cite{KR16}, \cite{KPR15} and \cite{KR18}. Below, we briefly describe a couple of other examples.
	
	\subsection*{Definable components in classical topological dynamics}
	
	In \cite{KP16}, the authors consider a topological group $G=G(M)$, definable in a structure $M$ with predicates for all open subsets of $G$, denoting $G(\fC)$ by $G^*$. They denote by $\mu$ the subgroup of $G^*$ of infinitesimal elements, that is, $\bigcap_U U(\fC)$, where $U$ ranges over all neighbourhoods of the identity in $G$. Using $\mu$, they define the group $G^{*000}_{\topo}$ as the smallest $M$-invariant normal subgroup of $G^*$ which contains $\mu$ and has bounded index. Then $G^*/G^{*000}_\topo$ is a new invariant of the topological group $G$ (as one can show that it does not depend on the choice of the model $M$, as long as it defines $G$ and has predicates for all its open subsets).
	
	They also define the space $S_{G^*}^\mu(M)$ as the quotient of $S_{G^*}(M)$ by $\mu$ (i.e.\ two types $p,q\in S_{G^*}(M)$ are identified if $\mu p(\fC)=\mu q(\fC)$).
	
	Then, since $G^*$ is definable, $G(M)$ is dense in $S_{G^*}(M)$, and so it is also dense in $S_{G^*}^\mu(M)$, so $(G(M),S_{G^*}^\mu(M),\tp(e/M))$ is an ambit. In fact, it is exactly the classical universal (topological) $G$-ambit.  It turns out that $S_{G^*}^\mu(M)$ has a natural semigroup structure which makes it isomorphic to $E(G(M),S_{G^*}^\mu(M))$, so in particular, we can find inside the Ellis group $u\cM$ and the quotient $u\cM/H(u\cM)$, which is exactly the generalized Bohr compactification of $G$, as introduced by Glasner in \cite[Chapter VIII]{Gl76}. They turn to state, in \cite[Theorem 2.24, Theorem 2.25]{KP16} (without proof, beyond very broad description how one can adapt \cite{KP17}) that we have a well-defined topological quotient map $u\cM/H(u\cM)\to G^*/G^{*000}_\topo$, and that $G^{*00}_\topo/G^{*000}_\topo$ is the quotient of a compact Hausdorff group by a dense subgroup, for $G^{*00}_\topo$ defined analogously.
	
	It is not hard to show that, in their context, the coset equivalence relation $E_{G^{*000}_\topo}$ of $G^{*000}_\topo$ induces a uniformly properly group-like $F_\sigma$ equivalence relation on the ambit $(G(M),S_{G^*}^\mu(M),\tp(e/M))$, and the quotient of $S_{G^*}^\mu(M)$ by this relation can be naturally identified with $G^*/G^{*000}_\topo$. Thus, by Lemma~\ref{lem:main_abstract_grouplike}, we recover the quotient map $u\cM/H(u\cM)\to G^*/G^{*000}_\topo$, concluding (using Lemma~\ref{lem:new_preservation_E_to_H}) that $G/G^{*000}_\topo$ is the quotient of the compact group $u\cM/H(u\cM)$ by an $F_\sigma$ normal subgroup. Since $G^{*00}_\topo/G^{*000}_\topo$ is the closure of the identity in $G^*/G^{*000}_\topo$, it follows that it is the quotient of a compact Hausdorff group by a dense subgroup.
	\subsection*{Relative Galois groups}
	\newcommand{\res}{{\mathrm{res}}}
	\newcommand{\fix}{{\mathrm{fix}}}
	In \cite{DKL17}, the authors study several variants of the Galois group. For each partial type $\Sigma$ over $\emptyset$, they put:
	\begin{itemize}
		\item
		$\Aut(\Sigma(\fC))=\{\sigma\restr_{\Sigma(\fC)}\mid \sigma\in \Aut(\fC) \}$,
		\item
		$\Autf_\res(\Sigma(\fC))=\{\sigma\restr_{\Sigma(\fC)}\mid \sigma\in \Autf(\fC) \}$,
		\item
		$\Autf_\fix(\Sigma(\fC))=\{\sigma\in \Aut(\Sigma(\fC))\mid \sigma(a)\equiv_\Lasc a\}$, where $a$ is a tuple enumerating $\Sigma(\fC)$.
	\end{itemize}
	Using these, they define the relative Galois groups in the following way.
	\begin{itemize}
		\item
		$\Gal^\res(\Sigma(\fC))=\Aut(\Sigma(\fC))/\Autf_\res(\Sigma(\fC))$
		\item
		$\Gal^\fix(\Sigma(\fC))=\Aut(\Sigma(\fC))/\Autf_\fix(\Sigma(\fC))$
	\end{itemize}
	It is easy to see that $\Autf_\res(\Sigma(\fC))$ and $\Autf_\fix(\Sigma(\fC))$ are normal subgroups of $\Aut(\Sigma(\fC))$ and $\Autf_\res(\Sigma(\fC))\leq \Autf_\fix(\Sigma(\fC))$, so $\Gal^\res(\Sigma(\fC))$ and $\Gal^\fix(\Sigma(\fC))$ are groups and we have a natural epimorphism $\Gal^\res(\Sigma(\fC))\to \Gal^\fix(\Sigma(\fC))$. Furthermore, the restriction epimorphism $\Aut(\fC)\to \Aut(\Sigma(\fC))$ induces an epimorphism $\Gal(T)\to \Gal^\res(\Sigma(\fC))$, which turns $\Gal^\res(\Sigma(\fC))$ and $\Gal^\fix(\Sigma(\fC))$ into topological groups. Furthermore, by considering the compositions of the epimorphisms with the function $S_m(M)\to \Gal(T)$ from Fact~\ref{fct:sm_to_gal}, we can also conclude that each relative Galois group also has a well-defined Borel cardinality (provided the theory is countable and $\Sigma$ has countably many free variables).
	
	In both cases, we can show that the Galois groups are actually quotients of the space $S_m(M)$ by a uniformly properly group-like equivalence relation. Thus, we can apply Lemma~\ref{lem:weakly_grouplike} to present them as quotients of compact Hausdorff groups, and if the language is countable, we may also apply Theorem~\ref{thm:main_abstract} to present them as quotients of compact Polish groups, and Corollary~\ref{cor:metr_smt_cls} to see that they are smooth if and only if they are Hausdorff  (i.e.\ they coincide with the appropriately defined relative Kim-Pillay Galois groups). Similarly to Theorem~\ref{thm:main_galois}, we also recover the full Borel cardinality under NIP, although in this case, it is enough to assume NIP on $[a]_{\equiv}$ for a suitable tuple of realisations of $\Sigma$.
	
	
	\section{Examples}
	\label{sec:examples}
	
	In this section, we analyse examples of non-G-compact theories $T$ from \cite{CLPZ01} and \cite{KPS13} and see how Theorem~\ref{thm:main_galois} can be applied to them. Namely, we describe the compact group $\hat G$ (which turns out to be the Ellis group) and the kernel of $\hat r\colon \hat G\to \Gal(T)$ in those cases. In order to do that, we compute the Ellis groups of the appropriate dynamical systems. This allows us to describe the group $\Gal(T)$ in each of these examples. Further, because the examples have NIP, this description also yields the Borel cardinality of the Galois group.
	
	The contents of this section are based on the appendix of \cite{KR18} (joint with Krzysztof Krupiński), expanded with more details of the proofs.
	
	(The topological group structure in the first example (Example~\ref{ex:CLPZ}) was described in \cite{Zie02}, by a more direct method. In \cite{KPS13}, the authors describe the topological group structure the second example (Example~\ref{ex:KPS}) and the Borel cardinality in both cases, but use completely different methods and give very few details.)
	
	\subsection*{Lemmas}
	First, we prove some auxiliary lemmas.
	
	\begin{rem}
		\label{rem:action_factors}
		If $(G,X)$ is a dynamical system and the action of $G$ on $X$ factors through another group $G'$, then it is easy to see that $E(G,X)=E(G',X)$, and the $\tau$ topologies on the ideal groups coincide.\xqed{\lozenge}
	\end{rem}
	
	
	\begin{lem}
		\label{lem:projlim_ellis}
		Consider a projective system of dynamical systems $(G_i,X_i)$ for $i\in I$ (where $i$ is some downwards directed set), i.e.\ for each pair $i<j$ we have an epimorphism $\pi_{i,j}\colon G_i\to G_j$, and a $G_i$-equivariant continuous surjection $\pi_{i,j}\colon X_i\to X_j$. Let $G:=\varprojlim_i G_i$ act naturally on $X:=\varprojlim_i X_i$, and for each $i$, let $\pi_i$ denote the projection $\pi_i\colon X\to X_i$ and abusing the notation, also the projection $\pi_i\colon G\to G_i$.
		
		Then we have a natural isomorphism $E(G,X)\cong \varprojlim_i E(G_i,X_i)$ (as semitopological semigroups and as a $G$-flows), consistent with the maps $\pi_i$ given in the preceding paragraph. Let us abuse the notation further, and write $\pi_i$ for the epimorphism $E(G,X)\to E(G_i,X_i)$.
		
		Then, for every minimal left ideal $\cM \unlhd E(G,X)$, each $\cM_i=\pi_i[\cM]$ is a minimal left ideal in $E(G_i,X_i)$ and $\cM=\varprojlim_i \pi_i[\cM]$. If $u\in \cM$ is an idempotent, then each $u_i=\pi_i(u)$ is an idempotent in $\cM_i=\pi_i[\cM]$.
		
		In particular, $u\cM=\varprojlim_i u_i\cM_i$. Furthermore, the $\tau$ topology on $u\cM$ is the projective limit topology, with each $u_i\cM_i$ equipped with its $\tau$ topology.
		
		Conversely, if $(\cM_i)_i$ is a consistent system of minimal left ideals in $E(G_i,X_i)$ and for each $i$, $u_i$ is an idempotent in $\cM_i$, then $\varprojlim_i \cM_i$ is a minimal left ideal in $E(G,X)$ and $u=(u_i)_i$ is an idempotent in $\cM$.
	\end{lem}
	\begin{proof}
		Note that immediately by the assumptions, $G$ acts on each $X_i$ via $G_i$, and $\pi_i\colon X\to X_i$ is $G$-equivariant. By Remark~\ref{rem:action_factors}, we may assume without loss of generality that $G=G_i$ for all $i$.
		
		Now, using Proposition~\ref{prop:induced_epimorphism}, we obtain the epimorphisms $\pi_i\colon E(G,X)\to E(G,X_i)$, and they obviously commute with the covering maps in the projective system, whence $E(G,X)=\varprojlim_i E(G,X_i)$.
		
		Since each $\pi_i$ is an epimorphism, preimages and images of ideals by $\pi_i$ are ideals. This easily implies that if $\cM$ is a minimal ideal, then each $\pi_i[\cM]$ is also minimal. Since $\cM$ is closed (as a minimal ideal), it is the inverse limit of $\pi_i[\cM]$. Conversely, if $(\cM_i)_i$ is a consistent system of minimal ideals, then $\varprojlim_i \cM_i=\bigcap_i \pi_i^{-1}[\cM_i]$, so it is an ideal (as an intersection of ideals, which is nonempty by compactness, because the system is consistent). Thus, it contains a minimal ideal $\cM$. If $\cM\subsetneq \varprojlim_i \cM_i$, then for some $i$ we have $\pi_i[\cM]\subsetneq \cM_i$, which contradicts minimality of $\cM_i$, so $\cM= \varprojlim_i \cM_i$, and the latter is a minimal ideal.
		
		It is clear that $u\in E(G,X)$ is an idempotent if and only if each $\pi_i(u)$ is an idempotent (because multiplication in the inverse limit is coordinatewise). Therefore, if $u=(u_i)_i\in \cM=\varprojlim_i \cM_i$ is an idempotent, then for each $f\in u\cM$ and each $i$ we have $\pi_i(f)=\pi_i(uf)=\pi_i(u)\pi_i(f)\in u_i \cM_i$, and conversely, if for each $i$ we have $\pi_i(f)\in u_i\cM_i$, then $uf=(u_i\pi_i(f))_i=(\pi_i(f))_i=f$, so as sets, $u\cM=\varprojlim_i u_i\cM_i$.
		
		What is left is to show that the $\tau$-topology on an Ellis group $u\cM$ is the limit of the $\tau$ topologies on projections. Let us denote the limit topology by $\pi$.
		
		In one direction, this is trivial: a subbasic $\pi$-closed set is clearly $\tau$-closed, so $\tau$ refines $\pi$.
		
		In the other direction, let $A$ be a $\tau$-closed set in $u\cM$. Take any $f$ which is in $\pi$-closure of $A$, any open $U\ni u$ and $V\ni f$, with the aim to apply Proposition~\ref{prop:circ_description} to show that $f\in u\circ A$. Then for some $i\in I$ and open $U',V'\subseteq EL_i=E(G,X_i)$, we have $U=\pi_i^{-1}[U'],V=\pi_i^{-1}[V']$, so $u_i\in U'$ and $f_i:=\pi_i(f)\in V'$. But then by the assumption and Proposition~\ref{prop:circ_description}, there is some $g_i\in G$ and $a_i\in \pi[A]$ such that $\pi_{X_i,g_i}\in U'$ and $g_ia_i\in V'$. But then for any $a\in A$ such that $\pi_i(a)=a_i$ we have $\pi_{X,g_i}\in U$ and $g_ia\in V$. Since $U,V$ were arbitrary, by Proposition~\ref{prop:circ_description}, $f\in u\circ A$, and since $f\in u\cM$ and $A$ is $\tau$-closed, we have $f\in A$.
	\end{proof}
	
	\begin{lem}
		\label{lem:tau_accumulation}
		Fix arbitrary dynamical system $(G,X)$, and consider its Ellis group $u\cM$.
		
		Given a net $(f_i)_i$ in $u\cM$ and $f\in u\cM$, the following are equivalent:
		\begin{itemize}
			\item
			$f$ is a $\tau$-accumulation point of $(f_i)_i$ (i.e.\ for every $i_0$, $f$ is in the $\tau$-closure of $(f_i)_{i>i_0}$),
			\item
			there is a subnet $(f'_{j'})_{j'}$ of $(f_j)_j$ such that for some net $(g_{j'})_{j'}$ in $G$ such that $g_{j'}\to u$ we have $g_{j'}f'_{j'}\to f$.
		\end{itemize}
		In particular, $(f_i)_i$ $\tau$-converges to $f$ if and only if every subnet of $(f_i)_i$ has a further subnet with the second property.
	\end{lem}
	\begin{proof}
		It is clear that the second condition implies the first. For the converse, by Proposition~\ref{prop:circ_description}, it is enough to show that for every $i_0$, for every open $U\ni u$ and $V\ni f$, there is some $i>i_0$ and $g_i\in G$ such that $g_i\in U$ and $g_if_i\in V$. But by the assumption, we can find some net $(g'_j)_j$ and a net $(f'_j)_j$, where each $f'_j\in f_{>{i_0}}$, such that $g'_j\to u$ and $g'_jf'_j\to f$. But then for sufficiently large $j$ we have $g'_j\in U$ and $g'_jf'_j\in V$, so we can just take any $i>i_0$ such that $f_i=f'_j$ and $g_i=g'_j$.
		
		The ``in particular" follows easily, as $(f_i)_i$ converges to $f$ exactly when $f$ is the accumulation point of every subnet of $(f_i)_i$.
	\end{proof}
	
	
	\begin{lem}
		\label{lem:product_ellis}
		Consider dynamical systems $(G_i,X_i)$ for $i\in I$ (where $I$ is some index set). Put $G=\prod_i G_i$ acting naturally on $X=\prod_i X_i$, and for each $i$, let $\pi_i$ be the projection $X\to X_i$ and, abusing the notation, $G\to G_i$.
		
		Then $E(G,X)\cong \prod_i E(G_i,X_i)$ (as a semitopological semigroup and as a $G$-flow).
		
		Furthermore, if $\cM$ is a minimal left ideal in $E(G,X)$, then each $\cM_i=\pi_i[\cM]$ is a minimal left ideal in $E(G_i,X_i)$ and $\cM=\prod_i \pi_i[\cM]$. If $u\in \cM$ is an idempotent, then each $u_i=\pi_i(u)$ is an idempotent in $\cM_i=\pi_i[\cM]$.
		
		In particular, $u\cM=\prod_i u_i\cM_i$. Furthermore, the $\tau$ topology on $u_i\cM_i$ is the product topology.
		
		Conversely, if $\cM_i$ is a minimal left ideal in $E(G_i,X_i)$ and $u_i$ is an idempotent in $\cM_i$, then $\prod_i \cM_i$ is a minimal left ideal in $E(G,X)$ and $u=(u_i)_i$ is an idempotent in $\cM$.
	\end{lem}
	\begin{proof}
		Since every product is the inverse limit of its finite subproducts, by Lemma~\ref{lem:projlim_ellis}, it is enough to consider the case when $I$ is finite. Moreover, a straightforward inductive argument shows that the case of finite products follows from the case of products of two elements.
		
		Thus, we may assume that $G=G_1\times G_2$ and $X=X_1\times X_2$. The fact that $E(G,X)$ is the product $E(G_1,X_1)\times E(G_2,X_2)$ is straightforward, as is the fact that minimal ideals in $E(G,X)$ are exactly the products of minimal ideals in $E(G_i,X_i)$, and that idempotents are those elements which have idempotents on both coordinates.
		
		The only nontrivial statement is about the $\tau$ topology being equal to the product topology. As in the case of inverse limit, let us call the latter topology $\pi$. Also as there, we see immediately that subbasic $\pi$-closed sets are $\tau$-closed, so $\tau$ refines $\pi$.
		
		In the other direction, consider any $A\subseteq u\cM=u_1\cM_1\times u_2\cM_2$ and let $f$ be a point in the $\pi$-closure of $A$. We will show that $f$ is also in the $\tau$-closure of $A$. We have a net $(a_i)_i$ in $A$ which is $\pi$-convergent to $f$, i.e.\ for $j=1,2$ we have $(a_{j,i})\xrightarrow{\tau} f_j$, where $f=(f_1,f_2)$ and each $a_i=(a_{1,i},a_{2,i})$.
		
		By applying Lemma~\ref{lem:tau_accumulation} to $(a_{1,i})_i$, we may assume without loss of generality that there is a net $(g_{1,i})_i$ in $G_1$ such that $g_{1,i}\to u_1$ and $g_{1,i}a_{1,i}\to f_1$. By applying it again, we may assume without loss of generality that there is also a net $(g_{2,i})_i$ in $G_2$ such that $g_{2,i}\to u_2$ and $g_{2,i}a_{2,i}\to f_2$. But then $(g_{1,i},g_{2,i})\to (u_1,u_2)=u$ and $(g_{1,i},g_{2,i})(a_{1,i},a_{2,i})\to (f_1,f_2)=f$, so $f$ is in $\tau$-closure of $A$, and we are done.
	\end{proof}
	
	
	\begin{prop}\label{prop:product of Ellis groups}
		Suppose we have a multi-sorted structure $M=(M_n)_n$, where the sorts $M_n$ are arbitrary, without any functions or relations between them. Enumerate each $M_n$ by $m_n$ and put $m=(m_n)_n$. Then $E(\Aut(M),S_m(M))\cong \prod_n E(\Aut(M_n),S_{m_n}(M))$, and similarly, the minimal left ideals and the Ellis groups (equipped with the $\tau$-topology) are the products of minimal left ideals and Ellis groups, respectively.
	\end{prop}
	\begin{proof}
		Under the given assumptions, it is easy to see that $\Aut(M)=\prod_n \Aut(M_n)$ and $S_m(M)=\prod_n S_{m_n}(M_n)$. The proposition follows from Lemma~\ref{lem:product_ellis}
	\end{proof}
	
	
	\subsection*{Examples}
	
	In this section, unless otherwise stated, $M_n$ denotes the countable structure $(M_n,R_n,C_n)$, where $n>1$ is a fixed natural number, the underlying set is ${\bQ}/{\bZ}$, $R_n$ is the unary function $x\mapsto x+1/n$, and $C_n$ is the ternary predicate for the natural (dense, strict) circular order. Let a tuple $m_n$ enumerate $M_n$. It is easy to show (see \cite[Proposition 4.2]{CLPZ01}) that $\Th(M_n)$ has quantifier elimination and the real circle $S^1_n=\bR/\bZ$ equipped with the rotation by the angle $2\pi/n$ and the circular order is an elementary extension of $M_n$. As usual, $\fC \succ S^1_n$ is a monster model.
	
	Given any $c' \in \fC$, by $\st(c')$ we denote the standard part of $c'$ computed in the circle $S^1=\bR/\bZ$.
	As $\st(c')$ depends only on $\tp(c'/M_n)$, this extends to a standard part mapping on the space of 1-types $S_1(M_n)$.
	
	\begin{prop}
		\label{prop:group_onetypes}
		If $u$ is an idempotent in a minimal left ideal $\cM$ of  the Ellis semigroup $E(\Aut(M_n),S_1(M_n))$, then $u\cM$ is generated by $R_nu$ and cyclic of order $n$. In particular, it is isomorphic to ${\bZ}/n{\bZ}$.
	\end{prop}
	
	\begin{proof}
		Note that $R_n$ is a $\emptyset$-definable automorphism of $M_n$, and as such, it is in the centre of $\Aut(M_n)$, and so it is also central in the Ellis semigroup.
		
		Now, since for any two Ellis groups $u\cM,v\cN$, the map $f\mapsto vfv$ is an isomorphism $u\cM\to v\cN$ (cf.\ Remark~\ref{rem:explicit_ellisgroup_isomorphism}), and since $R_n$ is central, we have $vR_n^juv=R_n^jvuv=R_n^jv$. Thus if the conclusion holds for $u\in \cM$, then it also holds for $v
		\in \cN$. Hence, it is enough to show that it holds for \emph{some} idempotent in \emph{some} minimal ideal.
		
		In the rest of this proof, by \emph{short} interval we mean an interval of length less than $1/n$. We also identify $\Aut(M_n)$ with its image in the Ellis semigroup.
		
		From quantifier elimination, it follows easily that $M_n$ is $\omega$-categorical, and $\Aut(M_n)$ acts transitively on the set of short open intervals in $M_n$.
		
		Denote by $J$ the set of $p\in S_1(M_n)$ with $\st(p)\in [0,1/n) + \bZ \subseteq \bR/\bZ$.
		
		\begin{clm*}
			For any non-isolated type $p\in S_1(M_n)$, there is a unique $f_{p}\in EL:=E(\Aut(M_n),S_1(M_n))$ such that for all $q\in J$ we have $f_p(q)=p$.
		\end{clm*}
		\begin{clmproof}
			Enumerate $M_n$ as $(a_k)_{k\in \bN}$.
			
			Since $p$ is non-isolated, for each $k\in \bN$ there is a short open interval $I_k$ such that $p$ is concentrated on $I_k$ and $a_0,\ldots,a_k\notin I_k$. By quantifier elimination, it is easy to see that $p$ is the only type in $S_1(M_n)$ concentrated on all $I_k$'s.
			
			Now, let $J_k:=(\frac{-1}{2kn},\frac{1}{n}-\frac{1}{kn})$. Notice that if $q\in J$, then $q$ is concentrated on all but finitely many $J_k$'s.
			
			Since each $I_k$ and $J_k$ is a short open interval, we can find for each $k$ some $\sigma_k\in \Aut(M_n)$ such that $\sigma_k[J_k]=I_k$. It follows that for any $q\in J$ we have $\lim_k\sigma_k(q)=p$. Thus, if we take any $f_p\in EL$ which is an accumulation point of $(\sigma_k)_k$, we will have $f_p(q)=p$ for all $q \in J$.
			
			To see that $f_p$ is unique, note that for each integer $j$ and $q\in R_n^j[J]$, $f_p(q)\in f_p[R_n^j[J]]=f_pR_n^j[J]=R_n^j f_p[J]=\{R_n^j(p)\}$. Since $J\cup R_n[J]\cup\ldots\cup R_n^{n-1}[J]=S_1(M_n)$, uniqueness follows.
		\end{clmproof}
		
		Take any non-isolated $p_0\in J$, and let $u=f_{p_0}$ (as in the claim). By uniqueness in the claim, $u$ is an idempotent. Denote by $\mathcal O$ the $R_n$-orbit of $p_0$.
		
		Note that every $f \in ELu$ is constant on $J$. As in the above proof of uniqueness, since $u$ and $uf$ commute with $R_n$, we easily see that the image of $uf$ equals $\mathcal O$.
		
		Now, we show that $\cM:=ELu$ is a minimal left ideal. Consider any $f \in \cM$.
		By the last paragraph, $uf(p_0)=R_n^j(p_0)$ for some $j$. Then $R_n^{-j}uf(p_0)=p_0$ and $R_n^{-j}uf$ is constant on $J$, so by uniqueness in the claim, $R_n^{-j}uf=u$. It follows that $ELf=ELu=\cM$, so $\cM$ is a minimal left ideal.
		
		
		By the preceding paragraph, we see also that for any $uf\in u\cM$, there is some $j$ such that $uf=R_n^ju$. Conversely, since $R_n$ is central, $R_nu=uR_nu\in u\cM$, so $u\cM$ is cyclic, generated by $R_nu$. As $R_n^ju(p_0)=R_n^j(p_0)$, $R_nu$ has order $n$ in $u\cM$, so $u\cM\cong \bZ/n\bZ$.
	\end{proof}
	
	\begin{lem}
		\label{lem:ellis_group_onesort}
		Suppose $n>1$.
		
		The restriction $S_{m_n}(M_n)\to S_1(M_n)$ to the first variable induces an isomorphism of Ellis semigroups $E(\Aut(M_n),S_{m_n}(M_n))\cong E(\Aut(M_n),S_1(M_n))$
		
		In particular, every Ellis group $u\cM$ of $(\Aut(M_n),S_{m_n}(M_n))$ is generated by $R_nu$ and isomorphic to $\bZ/n\bZ$.
	\end{lem}
	\begin{proof}
		We have the following ``orthogonality" property.
		
		\begin{clm*}
			Let $p,q\in S_{m_n}(M_n)$ satisfy the condition that for each single variable $x$, $p\restr_x=q\restr_x$. Then $p=q$.
		\end{clm*}
		\begin{clmproof}
			For $c_1',c_2'\in \fC$, write $c_1'<c_2'$ for $C_n(c_1',c_2',R_n(c_1'))$. Note that for each $r\in S^1$, this is a linear ordering on the set of all $c'$ with $\st(c')=r$. Furthermore, for any $c_1',c_2',c_3'$ we have that $C_n(c_1',c_2',c_3')$ holds if and only if one of the following holds:
			\begin{itemize}
				\item
				$\st(c_1'),\st(c_2'),\st(c_3')$ are all distinct and they are in the standard circular order on $S^1$,
				\item
				$\st(c_1')=\st(c_2')\neq \st(c_3')$ and $c_1'<c_2'$,
				\item
				$\st(c_1')\neq\st(c_2')=\st(c_3')$ and $c_2'<c_3'$,
				\item
				$\st(c_2')\neq\st(c_1')=\st(c_3')$ and $c_1'>c_3'$,
				\item
				$\st(c_1')=\st(c_2')=\st(c_3')$ and ($c_1'<c_2'<c_3'$ or $c_3'<c_1'<c_2'$ or $c_2'<c_3'<c_1'$).
			\end{itemize}
			
			We need to show that for each tuple $m'=(m'_k)_{k\in \bN}$ satisfying $\tp(m_n/\emptyset)$, we have the implication $\tp(m_n/\emptyset)\cup\bigcup_k \tp(m'_k/M_n)\vdash \tp(m'/M_n)$. By quantifier elimination, it is enough to show that the type on the left implies each atomic formula (or negation) in $\tp(m'/M_n)$. The only nontrivial cases are of the form $C_n(R_n^i(x_1),R_n^j(x_2),c)$, $C_n(R_n^i(x_1),c,R_n^j(x_2))$, $C_n(c,R_n^i(x_1),R_n^j(x_2))$ (or negations), where $i,j \in \{0,\dots,n-1\}$ and $c \in M_n$. But that follows immediately from the preceding paragraph (and the fact that the standard part is determined by the type over $M_n$).
		\end{clmproof}
		
		It follows from quantifier elimination that there is a unique 1-type over $\emptyset$, so the restriction to the first variable $S_{m_n}(M_n)\to S_1(M_n)$ is surjective, and (since it is obviously equivariant) it gives us a surjective homomorphism $E(\Aut(M_n),S_{m_n}(M_n))\to E(\Aut(M_n),S_1(M_n))$. We need to show that it is injective.
		
		Suppose $f_1,f_2\in E(\Aut(M_n),S_{m_n}(M_n))$ are distinct, so there is some $p\in S_{m_n}(M_n)$ such that $f_1(p)\neq f_2(p)$. But then, by the claim, there is a variable $x_k$ such that $f_1(p)\restr x_k\neq f_2(p)\restr x_k$. Choose $m'=(m'_k)_{k\in \bN}\models p$; then $m'$ enumerates a countable $M'\preceq \fC$. By $\omega$-categoricity and the fact that there is a unique 1-type over $\emptyset$, there is $\sigma\in \Aut(M')$ such that $\sigma(m'_1)=m'_k$. Now, if we put $p':=\tp(\sigma(m')/M_n)$, we have that $p'\restr_{x_1}=p\restr_{x_k}$. From that, we obtain $f_1(p')\restr_{x_1}=f_1(p)\restr_{x_k}\neq f_2(p)\restr_{x_k}=f_2(p')\restr_{x_1}$. It follows that the epimorphism $E(\Aut(M_n),S_{m_n}(M_n))\to E(\Aut(M_n),S_1(M_n))$ induced by the restriction to the first variable is injective.
		
		To complete the proof, notice that because restriction to $S_1(M_n)$ induces an isomorphism of $E(\Aut(M_n),S_{m_n}(M_n))$ and $E(\Aut(M_n),S_1(M_n))$, for any Ellis group $u\cM$ in the former, $u\cM\restr_{S_1(M_n)}$ is an Ellis group in the latter. Furthermore, it is easy to see that $\pi_{R_n,S_{m_n}(M_n)}\restr_{S_1(M_n)}=\pi_{R_n,S_{1}(M_n)}$, so since --- by Proposition~\ref{prop:group_onetypes} --- $u\cM\restr_{S_1(M_n)}$ is cyclic of order $n$, generated by $R_nu\restr_{S_1(M_n)}=\pi_{R_n,S_{1}(M_n)}u\restr_{S_1(M_n)}$ , it follows that $u\cM$ is generated by $\pi_{R_n,S_{m_n}(M_n)}u=R_nu$.
	\end{proof}
	
	
	\begin{prop}
		\label{prop:connecting_maps_nn'}
		If $n,n'$ are positive integers and $n'$ divides $n$, then the map $M_n\to M_{n'}$ given by multiplication by $k=n/n'$ induces an epimorphism $\varphi_1\colon \Aut(M_n)\to\Aut(M_{n'})$ and a continuous surjection $\varphi_2\colon S_{m_n}(M_n)\to S_{m_{n'}}(M_{n'})$ which is equivariant with respect to the induced action of $\Aut(M_n)$ on $S_{m_{n'}}(M_{n'})$.
	\end{prop}
	\begin{proof}
		
		Denote by $H$ the group generated by $R_n^{n'}$ in $\Aut(M_n)$. Then the $H$-orbit equivalence relation on $M_n$ is definable in $M_n$, and thus $M_n/H$ is an imaginary sort in $M_n$. $R_n$ and $C_n$ induce an unary function $R_n'$ and a ternary relation $C_n'$ (both $M_n$-definable) on $M_n/H$ by putting $R_n'(Hx):=HR_n(x)$ and declaring that $C_n'(Hx_1,Hx_2,Hx_3)$ if we have $C_n(x_1',x_2',x_3')$ for the representatives $x_1',x_2',x_3'$ (of the respective orbits) in $[0,1/k)+\bZ$. Then it is not hard to see that $(M_n/H,C_n')$ is a dense circular order and $R_n'$ is its automorphism of order $n/k=n'$. Furthermore, it is easy to see that the map $M_n\to M_{n'}$ given by $x\mapsto kx$ factors through $Hx\mapsto kx$, which defines an isomorphism $\varphi_0\colon (M_n/H,R_n',C_n')\to(M_{n'},R_{n'},C_{n'})$.
		
		Then the isomorphism $\varphi_0$ induces an action of $\Aut(M_n)$ on $M_{n'}$ by automorphisms (given by $\sigma(x')=\varphi_0(\sigma(\varphi_0^{-1}(x')))$), and thus gives us a homomorphism $\varphi_1\colon \Aut(M_n)\to \Aut(M_{n'})$. Note that in particular, if $x'=kx$ for some $x\in M_n$, then $\varphi_1(\sigma)(kx)=\varphi_0(\sigma(Hx))=\varphi_0(H\sigma(x))=k\sigma(x)$. Since $x\mapsto kx$ is onto $M_{n'}$, this determines $\varphi_1(\sigma)$ uniquely, and in this sense $\varphi_1$ is induced by $x\mapsto kx$.
		
		Let $m_n$ be enumerated as $(m_n^i)_{i\in \bN}$. Then the natural map $[m_n]_{\equiv}\to [(Hm_n^i)_i]_{\equiv}$ (given by taking each coordinate to its $H$-orbit) is type-definable and induces a natural continuous surjection $S_{m_n}(M_n)\to S_{(Hm_n^i)_i}(M_n)$. Via the isomorphism between $(M_n/H,C_n',R_n')$ and $(M_{n'},C_{n'},R_{n'})$, we obtain a continuous surjection $S_{(Hm_n^i)_i}(M_n)\to S_{m_{n'}}(M_{n'})$. By composing the two, we obtain a surjection $\varphi_2\colon S_{m_n}(M_n)\to S_{m_{n'}}(M_{n'})$, which is easily seen to be $\Aut(M_n)$-equivariant, furthermore, it is not hard to see that it is induced by $x\mapsto kx$ in the sense that if $m_n'\equiv m_n$ is a tuple in $M_n$, then, the type $\tp(m_n'/M_n)$ is mapped to $\tp(km_n'/M_{n'})$ (note that since $M_n$ is $\omega$-categorical, realised types are dense in $S_{m_n}(M_n)$, so by continuity, this uniquely determines $\varphi_2$).
		
		What is left is to show that $\varphi_1$ is surjective. By the description of $\varphi_1$ given before, it is enough to show that for every $\sigma'\in \Aut(M_{n'})$ there is some $\sigma\in\Aut(M_n)$ such that for all $x\in M_{n}$, $k\sigma(x)=\sigma'(kx)$.
		
		We may assume without loss of generality that $\sigma'(0)=0$: otherwise, $\sigma'-\sigma'(0)$ is also an automorphism of $\Aut(M_{n'})$, and if we find $\sigma$ such that $k\sigma(x)=\sigma'(kx)-\sigma'(0)$, then for any $\alpha$ such that $k\alpha=\sigma'(0)$, $\sigma(x)+\alpha$ is an automorphism of $M_n$ and we have $k(\sigma(x)+\alpha)=\sigma'(kx)$.
		
		Put $I_j:=([j/k,(j+1)/k)\cap \bQ)+\bZ\subseteq \bQ/\bZ$ (where $j\in \bZ$). Then we can define $\sigma(x)$ for $x\in I_j$ by letting $\sigma(x)$ be the unique element of $I_j$ such that $k\sigma(x)=\sigma'(kx)$. We need to show that $\sigma$ is an automorphism, and then we will obviously have $\varphi_1(\sigma)=\sigma'$.
		
		Note that we can also describe $\sigma(x)$ in the following way: if $x\in I_j$ and $\beta\in [0,1)$ is a representative of $\sigma'(kx)$, then $\sigma(x)=(\beta+j)/k+\bZ$.
		
		It is not hard to see that for each $j$, $\sigma$ restricts to a bijection from $I_j$ to itself. In particular, $\sigma$ is a bijection. It is also not hard to see that it preserves the circular ordering (roughly, because it preserves the circular order between the chunks $I_j$, it preserves the circular order for triples where not all elements are in a single $I_j$, while for triples lying in a single $I_j$, it follows from the description of $C_n'$ and the fact that $\varphi_0$ is an isomorphism).
		
		Note that since $\sigma'(0)=0$, it follows that for any $x'\in M_{n'}$, $x'\in [1-1/n',1)+\bZ$ if and only if $\sigma'(x')\in [1-1/n',1)+\bZ$. Next, notice that for $x\in I_j$, we have that $kx\in [1-1/n',1)+\bZ$ if and only if $x\in [(j+1)/k-1/n,(j+1)/k)$, which (since $x\in I_j$) is equivalent to $x+1/n\in I_{j+1}$. Thus, for $x\in I_j$, we have $\sigma'(kx)\in [1-1/n',1)+\bZ$ if and only if $R_n(x)\in I_{j+1}$.
		
		It follows that $\sigma$ preserves $R_n$: fix any $x\in I_j$ and let $\beta\in [0,1)$ be a representative of $\sigma'(kx)$. Note that in this case $R_n(x)\in I_j$ or $R_n(x)\in I_{j+1}$. By the preceding paragraph it easily follows that either
		\begin{itemize}
			\item
			 $R_n(x)\in I_j$ and $\beta+1/n'\in [0,1)$, or
			\item
			$R_n(x)\in I_{j+1}$ and $\beta+1/n'-1\in [0,1)$.
		\end{itemize}
		Because $\sigma'$ is an automorphism of $M_{n'}$, it commutes with addition of $1/n'$, so
		\[
			\sigma'(k(x+1/n))=\sigma'(kx+1/n')=\sigma'(kx)+1/n'=(\beta+\bZ)+1/n',
		\]
		which shows that both $\beta+1/n'$ and $\beta+1/n'-1$ are representatives of $\sigma'(kR_n(x))$. Thus if $R_n(x)\in I_j$, then
		\[
			\sigma(R_n(x))=(\beta+1/n'+j)/k+\bZ=(\beta+j)/k+1/n+\bZ=\sigma(x)+1/n=R_n\sigma(x).
		\]
		Likewise, if $R_n(x)\in I_{j+1}$, then
		\[
			\sigma(R_n(x))=((\beta+1/n'-1)+(j+1))/k+\bZ=R_n\sigma(x).
		\]
		Thus, $\sigma$ is an automorphism such that $\varphi_1(\sigma)=\sigma'$, so $\varphi_1$ is onto, and we are done.
	\end{proof}


	\begin{cor}
		\label{cor:ellis_groups_allsorts}
		For all positive integers $n$, the Ellis group of $S_{m_n}(M_n)$ is isomorphic to $\bZ/n\bZ$, generated by $R_nu_n$, where $u_n$ is the identity in the Ellis group.
	\end{cor}
	\begin{proof}
		For $n\neq 1$, this is Lemma~\ref{lem:ellis_group_onesort}, so we only need to consider $n=1$.
		
		By Proposition~\ref{prop:connecting_maps_nn'}, Remark~\ref{rem:action_factors} and Proposition~\ref{prop:induced_epimorphism}, for each $n$, we have an epimorphism from the Ellis group of $S_{m_n}(M_n)$ onto the Ellis group of $S_{m_1}(M_1)$. In particular, since the Ellis groups of $S_{m_2}(M_2)$ and $S_{m_3}(M_3)$ are isomorphic to $\bZ/2\bZ$ and $\bZ/3\bZ$ respectively, the Ellis group of $S_{m_1}(M_1)$ is cyclic of order dividing $2$ and $3$. As such, it must be trivial.
		
		Note that $R_1$ is simply the identity, so $R_1u_1=u_1$ indeed generates the trivial group $\{u_1\}$.
	\end{proof}
	
	
	\begin{ex}
		\label{ex:CLPZ}
		Consider the theory $T$ of the multi-sorted structure $M=(M_n)_{n\in \bN^+}$, where each $M_n=(M_n,R_n,C_n)$ is the countable model as described at the beginning of this section. Then, if we enumerate $M$ as $m$, then $M$ is ambitious (because it is $\omega$-categorical). For each $n$, choose an idempotent $u_n'$ as i
		
		By Corollary~\ref{cor:ellis_groups_allsorts} and Proposition~\ref{prop:product of Ellis groups}, the Ellis group $u\cM$ of  the dynamical system $(\Aut(M),S_m(M))$ is isomorphic to $\prod_n \bZ/n\bZ$ with the product topology. Moreover, each element $f\in u\cM$ can be uniquely represented as $(R_n^{b_n}u_n)_n$, where $u_n$ is the restriction $u\restr_{S_{m_n}(M_n)}$, while $b_n$ is an integer in the interval $(-n/2,n/2]$. In particular, $u\cM$ is a Hausdorff (compact and Polish) group, so $H(u\cM)$ is trivial.
		
		Moreover, the group $D$ (i.e.\ $[u]_\equiv\cap u\cM$) is trivial. Indeed, if $f\in u\cM$ is nontrivial, then for some $n$, $f\restr_{S_{m_n}(M_n)}$ and $u\restr_{S_{m_n}(M_n)}$ are distinct. Therefore, $f\restr_{S_{m_n}(M_n)}=R_n^{b_n}(u\restr_{S_{m_n}(M_n)})$ for some $b_n$ not divisible by $n$, so in particular, $f(\tp(m_n/M_n))=R_n^{b_n}u(\tp(m_n/M_n))$, which is clearly distinct from $u(\tp(m_n/M_n))$. Hence, also $f(\tp(m/M))\neq u(\tp(m/M))$, i.e.\ $f\notin D$.
		
		We have proved that $u\cM/H(u\cM)D=u\cM/H(u\cM)=u\cM \cong \prod_n \bZ/n\bZ$, so the group $\hat G$ from Theorem~\ref{thm:main_galois} is $u\cM$, which we identify with $\prod_n \bZ/n\bZ$.
		
		We claim that $g\in \ker \hat r$ if and only if the $g_n$'s are absolutely bounded.
		
		By \cite[Corollary~4.3]{CLPZ01}, for any $a\in M_n(\fC)$ and integer $k \in (-n/2,n/2]$ we have $d_\Lasc(a,R_n^k(a)) \geq k$, which easily implies (having in mind the precise identification of $u\cM$ with $\prod_n \bZ/n\bZ$) that unbounded sequences are not in the kernel.
		
		On the other hand, to show that absolutely bounded sequences are in $\ker \hat r$, it is enough to show this for sequences bounded by $1$. But then (again, having in mind the identification of $u\cM$ with $\prod_n \bZ/n\bZ$ from the second paragraph) for an element $f \in u\cM$ corresponding to such a sequence, $f\restr_{S_{m_n}(M_n)} =R_n^{\epsilon_n}u \restr_{S_{m_n}(M_n)}=u \restr_{S_{m_n}(M_n)}R_n^{\epsilon_n}$ for some $\epsilon_n \in \{-1,0,1\}$. By \cite[Lemma 3.7]{CLPZ01}, it is enough to show that $d_\Lasc(m_n,R_n(m_n))$ is bounded (when $n$ varies).
		%
		By $\omega$-categoricity, we can replace $m_n$ by an enumeration $m_n'$ of any other countable model $M_n'$. So let $m'_n$ be an enumeration of $(\bQ\cap ([0,1/3n)+\bZ/n ))/\bZ\subseteq \bQ/\bZ$. Furthermore, put $m''_n:=m'_n+1/3n$ and $m'''_n:=m'_n+2/3n$, and write $M'_n,M''_n,M'''_n$ for the respective models they enumerate. Then $\tp(m'_n/M'''_n)=\tp(m''_n/M'''_n)$, $\tp(m''_n/M'_n)=\tp(m'''_n/M'_n)$, $\tp(m'''_n/M''_n)=\tp(R_n(m'_n)/M''_n)$, so $d_\Lasc(m_n',R_n(m_n')) \leq 3$.
		
		Note that $T$ has NIP (e.g.\ because it is interpretable in an o-minimal theory), so the full Theorem~\ref{thm:main_galois} applies, and the Galois group $\Gal(T)$ is the quotient of $\prod_n \bZ/n\bZ$ by the subgroup of bounded sequences. As a topological group, this is exactly the description given by \cite[Theorem~28]{Zie02}; note that the topology is trivial. In terms of Borel cardinality, we obtain $\ell^\infty$
		(see the paragraph following the proof of Lemma 3.10 in \cite{KPS13}).
		\xqed{\lozenge}
	\end{ex}
	
	\begin{ex}
		\label{ex:KPS}
		Consider the theory $T$ of the multi-sorted structure $M=(M_n,h_{nn'})_{n,n'}$, where $M_n$ are as before, $n$ runs over positive integers, while $n'$ rangers over divisors of $n$; for each pair $n' \mid n$, $h_{nn'}\colon M_n\to M_{n'}$ is the multiplication by $n/n'$. Enumerate each $M_n$ by $m_n$ in such a way that $h_{nn'}(m_{n})=m_{n'}$, and then enumerate $M$ by $m=(m_n)_n$.
		
		By Proposition~\ref{prop:connecting_maps_nn'}, we see that each map $h_{nn'}$ induces a natural epimorphism $\Aut(M_n)\to \Aut(M_{n'})$, and a continuous, $\Aut(M_n)$-equivariant surjection $S_{m_n}(M_n)\to S_{m_{n'}}(M_{n'})$, so we have epimorphisms of dynamical systems $(\Aut(M_n),S_{m_n}(M_n))\to (\Aut(M_{n'}),S_{m_{n'}}(M_{n'}))$. Furthermore, if $n''\mid n' \mid n$, then it is easy to see that $h_{n'n''}\circ h_{nn'}=h_{nn''}$, so these epimorphisms are compatible. Using that, it is not hard to see that $\Aut(M)=\varprojlim_n \Aut(M_n)$ (because $\Aut(M)$ acts on each $M_n$ by automorphisms and it has to be compatible with $h_{nn'}$) and $S_m(M)=\varprojlim_n S_{m_n}(M_n)$ (because the type on the $n$-th coordinate determines the type on $n'$-th coordinate, when $n'$ divides $n$). By Lemma~\ref{lem:projlim_ellis}, it follows that $E(\Aut(M),S_{m}(M))\cong \varprojlim_n E(\Aut(M_n),S_{m_n}(M_n))$.
		
		In particular, by Corollary~\ref{cor:ellis_groups_allsorts}, the Ellis group $u\cM$ of $E(\Aut(M),S_m(M))$ is isomorphic to the profinite completion of integers $\widehat \bZ=\varprojlim_n\bZ/n\bZ$. By analysis analogous to the preceding example, we see that $H(u\cM)$ and $D$ are trivial, and $\ker \hat r$ corresponds to the elements of $\widehat \bZ$ represented by bounded sequences. Those sequences are exactly the elements of $\bZ\subseteq \widehat \bZ$ (this follows from the observation that a bounded sequence representing an element of $\widehat \bZ$ has to eventually stabilize).
		
		Thus, by Theorem~\ref{thm:main_galois}, $\Gal(T)$ is the quotient $\widehat \bZ/\bZ$ (which, again, has trivial topology), and, since the theory is NIP (e.g.\ because, as it is easy to see, it is interpretable in $(\bR,+,\cdot,\leq)$), $\Gal(T)$ also has the Borel cardinality of $\widehat\bZ/\bZ$ which is $E_0$ (which can be seen as a consequence of the fact that it is hyperfinite (as an orbit equivalence relation of a $\bZ$-action) and non-smooth (as the quotient of a compact Polish group by a non-closed subgroup), see \cite[Theorem 8.1.1]{kanovei}.
		\xqed{\lozenge}
	\end{ex}
